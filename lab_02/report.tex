

\documentclass[12pt]{report}
\usepackage[utf8]{inputenc}
\usepackage[russian]{babel}
\usepackage[14pt]{extsizes}
\usepackage{listings}
\usepackage{graphicx}
\usepackage{amsmath,amsfonts,amssymb,amsthm,mathtools} 
\usepackage{pgfplots}
\usepackage{filecontents}
\usepackage{float}
\usepackage{indentfirst}
\usepackage{eucal}
\usepackage{enumitem}
%s\documentclass[openany]{book}
\frenchspacing

\usepackage{titlesec}
\titleformat{\section}
{\normalsize\bfseries}
{\thesection}
{1em}{}
\titlespacing*{\chapter}{0pt}{-30pt}{8pt}
\titlespacing*{\section}{\parindent}{*4}{*4}
\titlespacing*{\subsection}{\parindent}{*4}{*4}

\usepackage{indentfirst} % Красная строка

\usetikzlibrary{datavisualization}
\usetikzlibrary{datavisualization.formats.functions}

\usepackage{amsmath}


% Для листинга кода:
\lstset{ %
	language=c,                 % выбор языка для подсветки (здесь это С)
	basicstyle=\small\sffamily, % размер и начертание шрифта для подсветки кода
	numbers=left,               % где поставить нумерацию строк (слева\справа)
	numberstyle=\tiny,           % размер шрифта для номеров строк
	stepnumber=1,                   % размер шага между двумя номерами строк
	numbersep=5pt,                % как далеко отстоят номера строк от подсвечиваемого кода
	showspaces=false,            % показывать или нет пробелы специальными отступами
	showstringspaces=false,      % показывать или нет пробелы в строках
	showtabs=false,             % показывать или нет табуляцию в строках
	frame=single,              % рисовать рамку вокруг кода
	tabsize=2,                 % размер табуляции по умолчанию равен 2 пробелам
	captionpos=t,              % позиция заголовка вверху [t] или внизу [b] 
	breaklines=true,           % автоматически переносить строки (да\нет)
	breakatwhitespace=false, % переносить строки только если есть пробел
	escapeinside={\#*}{*)}   % если нужно добавить комментарии в коде
}


\usepackage[left=2cm,right=2cm, top=2cm,bottom=2cm,bindingoffset=0cm]{geometry}
% Для измененных титулов глав:
\usepackage{titlesec, blindtext, color} % подключаем нужные пакеты
\definecolor{gray75}{gray}{0.75} % определяем цвет
\newcommand{\hsp}{\hspace{20pt}} % длина линии в 20pt
% titleformat определяет стиль
\titleformat{\chapter}[hang]{\Huge\bfseries}{\thechapter\hsp\textcolor{gray75}{|}\hsp}{0pt}{\Huge\bfseries}


% plot
\usepackage{pgfplots}
\usepackage{filecontents}
\usetikzlibrary{datavisualization}
\usetikzlibrary{datavisualization.formats.functions}

\begin{document}
	%\def\chaptername{} % убирает "Глава"
	\thispagestyle{empty}
	\begin{titlepage}
		\noindent \begin{minipage}{0.15\textwidth}
			\includegraphics[width=\linewidth]{img/b_logo}
		\end{minipage}
		\noindent\begin{minipage}{0.9\textwidth}\centering
			\textbf{Министерство науки и высшего образования Российской Федерации}\\
			\textbf{Федеральное государственное бюджетное образовательное учреждение высшего образования}\\
			\textbf{~~~«Московский государственный технический университет имени Н.Э.~Баумана}\\
			\textbf{(национальный исследовательский университет)»}\\
			\textbf{(МГТУ им. Н.Э.~Баумана)}
		\end{minipage}
		
		\noindent\rule{18cm}{3pt}
		\newline\newline
		\noindent ФАКУЛЬТЕТ $\underline{\text{«Информатика и системы управления»}}$ \newline\newline
		\noindent КАФЕДРА $\underline{\text{«Программное обеспечение ЭВМ и информационные технологии»}}$\newline\newline\newline\newline\newline
		
		\begin{center}
			\noindent\begin{minipage}{1.1\textwidth}\centering
				\Large\textbf{  Отчет по лабораторной работе №2}\newline
				\textbf{по дисциплине <<Функциональное и логическое}\newline
				\textbf{~~~программирование>>}\newline\newline
			\end{minipage}
		\end{center}
		
		\noindent\textbf{Тема} $\underline{\text{Определение функций пользователя}}$\newline\newline
		\noindent\textbf{Студент} $\underline{\text{Зайцева А. А.~~~~~~~~~~~~~~~~~~~~~~~~~~~~~~~~~~~~~~~~~~}}$\newline\newline
		\noindent\textbf{Группа} $\underline{\text{ИУ7-62Б~~~~~~~~~~~~~~~~~~~~~~~~~~~~~~~~~~~~~~~~~~~~~~~~~~}}$\newline\newline
		\noindent\textbf{Оценка (баллы)} $\underline{\text{~~~~~~~~~~~~~~~~~~~~~~~~~~~~~~~~~~~~~~~~~~~~~~~~~}}$\newline\newline
		\noindent\textbf{Преподаватели} $\underline{\text{Толпинская Н.Б., Строганов Ю. В.~~~~~~~~~~~~~~~~~~~~~~~~~~~~}}$\newline\newline\newline
		
		\begin{center}
			\vfill
			Москва~---~\the\year
			~г.
		\end{center}
	\end{titlepage}
	
\chapter*{Теоретические вопросы}

\section*{1. Базис Lisp}
	
Базис -- это минимальный набор инструментов языка и стркутур данных, который позволяет решить любые задачи.


Базис Lisp :

\begin{itemize}
	\item атомы и структуры (представляющиеся бинарными узлами);
	\item базовые (несколько) функций и функционалов: встроенные — примитивные 
	функции (atom, eq, cons, car, cdr); специальные функции и функционалы (quote, cond, lambda, eval, apply, funcall).
	
\end{itemize}

Атомы:
\begin{itemize} 
	\item символы (идентификаторы) – синтаксически – набор литер (букв и цифр), начинающихся с буквы;
	\item специальные символы – {T, Nil} (используются для обозначения логических констант);
	\item самоопределимые атомы – натуральные числа, дробные числа, вещественные числа, строки – последовательность символов, заключенных в двойные апострофы (например, “abc”);
\end{itemize} 

Более сложные данные – списки и точечные пары (структуры), которые строятся с помощью унифицированных структур – блоков памяти – бинарных узлов.

Определения:

Точечная пара ::= (<атом> . <атом>) | (<атом> . <точечная пара>) | (<точечная пара> . <атом>) | (<точечная пара> . <точечная пара>);

Список ::= <пустой список> | <непустой список>, где 

<пустой список> ::= () | Nil,

<непустой список> ::= (<первый элемент> . <хвост>),

<первый элемент> ::= <S-выражение>,

S-выражение ::= <атом> | <точечная пара>,

<хвост> ::= <список>.


Функцией называется правило, по которому каждому значению одного или нескольких  аргументов ставится в соответствие конкретное значение результата.

Функционалом, или функцией высшего порядка называется функция, аргументом или  результатом которой является другая функция.



	
\section*{2.Классификация функций}

Один из вариантов классификации функций:

\begin{itemize}
	\item чистые  математические функции (имеют фиксированное количество аргументов, сначала выяисляются все аргументы, а только потом к ним применяется функция);
	\item рекурсивные функции (основной способ выполнения повторных вычислений);
	\item специальные функции, или формы (могут принимать произвольное количество аргументов, или аргументы могут обрабатываться по-разному);
	\item псевдофункции (создают «эффект», например, вывод на экран);
	\item функции с вариантами значений, из которых выбирается одно;
	\item функции высших порядков, или функционалы --  функции, аргументом или  результатом которых является другая функция (используются для построения синтаксически управляемых программ);
\end{itemize}


\section*{3. Способы создания функций}






	
\chapter*{Практические задания}	

\section*{1. Составить диаграмму вычисления следующих выражений:
}

1. (equal 3 (abs - 3))

%\includegraphics[scale=1]{img/1.1}

2. (equal (+ 1 2) 3)

3. (equal (* 4 7) 21)

4. (equal (* 2 3) (+ 7 2))

5. (equal (- 7 3) (* 3 2))

6. (equal (abs (- 2 4)) 3)










3. Написать функцию (f ar1), возвращающую (((ar1))).
\begin{lstlisting}[language=Lisp]
	;;
	(defun f3 (ar1) (list (list (list ar1))))
	(f3 1) => (((1)))
	;;
	(lambda (ar1) (list (list (list ar1))))
	((lambda (ar1) (list (list (list ar1)))) 1) => (((1)))
\end{lstlisting}



	\bibliographystyle{utf8gost705u}  % стилевой файл для оформления по ГОСТу
	
	\bibliography{51-biblio}          % имя библиографической базы (bib-файла)
	
	
\end{document}
