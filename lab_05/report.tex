

\documentclass[12pt]{report}
\usepackage[utf8]{inputenc}
\usepackage[russian]{babel}
\usepackage[14pt]{extsizes}
\usepackage{listings}
\usepackage{graphicx}
\usepackage{amsmath,amsfonts,amssymb,amsthm,mathtools} 
\usepackage{pgfplots}
\usepackage{filecontents}
\usepackage{float}
\usepackage{indentfirst}
\usepackage{eucal}
\usepackage{enumitem}
%s\documentclass[openany]{book}
\frenchspacing

\usepackage{titlesec}
\titleformat{\section}
{\normalsize\bfseries}
{\thesection}
{1em}{}
\titlespacing*{\chapter}{0pt}{-30pt}{8pt}
\titlespacing*{\section}{\parindent}{*4}{*4}
\titlespacing*{\subsection}{\parindent}{*4}{*4}

\usepackage{indentfirst} % Красная строка

\usetikzlibrary{datavisualization}
\usetikzlibrary{datavisualization.formats.functions}
\long\def\comment{}
\usepackage{amsmath}

\usepackage{amssymb}

% Для листинга кода:
\lstset{ %
	language=lisp,                 % выбор языка для подсветки (здесь это С)
	texcl=true,
	extendedchars=\true,
	basicstyle=\small\sffamily, % размер и начертание шрифта для подсветки кода
	numbers=left,               % где поставить нумерацию строк (слева\справа)
	numberstyle=\tiny,           % размер шрифта для номеров строк
	stepnumber=1,                   % размер шага между двумя номерами строк
	numbersep=5pt,                % как далеко отстоят номера строк от подсвечиваемого кода
	showspaces=false,            % показывать или нет пробелы специальными отступами
	showstringspaces=false,      % показывать или нет пробелы в строках
	showtabs=false,             % показывать или нет табуляцию в строках
	frame=single,              % рисовать рамку вокруг кода
	tabsize=2,                 % размер табуляции по умолчанию равен 2 пробелам
	captionpos=t,              % позиция заголовка вверху [t] или внизу [b] 
	breaklines=true,           % автоматически переносить строки (да\нет)
	breakatwhitespace=false, % переносить строки только если есть пробел
	escapeinside={\#*}{*)},  % если нужно добавить комментарии в коде
	%inputencoding=utf8x,
	%extendedchars=\true
}



\usepackage[left=2cm,right=2cm, top=2cm,bottom=2cm,bindingoffset=0cm]{geometry}
% Для измененных титулов глав:
\usepackage{titlesec, blindtext, color} % подключаем нужные пакеты
\definecolor{gray75}{gray}{0.75} % определяем цвет
\newcommand{\hsp}{\hspace{20pt}} % длина линии в 20pt
% titleformat определяет стиль
\titleformat{\chapter}[hang]{\Huge\bfseries}{\thechapter\hsp\textcolor{gray75}{|}\hsp}{0pt}{\Huge\bfseries}


% plot
\usepackage{pgfplots}
\usepackage{filecontents}
\usetikzlibrary{datavisualization}
\usetikzlibrary{datavisualization.formats.functions}

\begin{document}
	%\def\chaptername{} % убирает "Глава"
	\thispagestyle{empty}


	




\chapter*{Практические задания}	

\section*{1. Написать функцию, которая по своему списку-аргументу lst определяет, является ли он палиндромом (то есть равны ли lst и (reverse lst)). Списки одноуровневые}


\begin{lstlisting}[language=Lisp]	
; 1. Проверка на равенство исходного списка и инвертированного исходного списка.

(defun is_palindrome (lst) 
	(equalp 
		lst 
		(reverse lst)
	)
)

; 2. Проверка на равенство первой половины исходного списка и 
; инвертированной второй половины исходного списка (если список нечетной длины, 
; то центральный элемент не попадает ни в первый, ни во второй список)
; (этот вариант и было предложено реализовать)

;(nthcdr n lst) выполняет для списка lst операцию cdr n раз, и возвращает результат
;(floor n) усекает значения по нижней границе
;(ceiling n) усекает значения по верхней границе.

(defun first_n (n lst) 
	(cond 
		((null lst) lst)
		((= n 0) Nil)
		(t (cons 
				(car lst) 
				(first_n (- n 1) (cdr lst))
			 )
		)
	)
)

(defun is_palindrome (lst)
	(let ((half_len (/ (length lst) 2)))
		(equalp 
			(first_n (floor half_len) lst) 
			(reverse (nthcdr (ceiling half_len) lst))
		)
	)
)



; 3. Рекурсивно: сравнить первый и последний элемент исходного списка, первый и последний 
; элемент исходного списка без первого и последнего элемента и так далее.
; (если длина списка нечетная, то центральный элемент ни с чем не сравнивается)

(defun list_without_last (lst)
	(cond
		((null (cdr lst)) Nil)
		(t (cons 
				(car lst) 
				(list_without_last (cdr lst))
			)
		)
	)
)

(defun is_palindrome (lst)
	(cond 
		((null (cdr lst)) t)
		((eql (car lst) (car (last lst))) ;т. к. (last '(1 2))=>(2)
			(is_palindrome (list_without_last (cdr lst))))
	)
)
\end{lstlisting}


Все варианты функций проверялись на следующих тестах:
\begin{lstlisting}[language=Lisp]
(is_palindrome Nil) => T
(is_palindrome '(1)) => T
(is_palindrome '(1 2 3)) => NIL
(is_palindrome '(1 2 1)) => T
(is_palindrome '(1 2 3 1)) => NIL
(is_palindrome '(1 2 2 1)) => T
\end{lstlisting}

\if 0
$
(is_palindrome Nil)
(is_palindrome '(1))
(is_palindrome '(1 2 3))
(is_palindrome '(1 2 1))
(is_palindrome '(1 2 3 1))
(is_palindrome '(1 2 2 1))
$
\fi



\clearpage
\section*{2. Написать предикат set-equal, который возвращает t, если два его множества-аргумента содержат одни и те же элементы, порядок которых не имеет значения}

Все элементы первого множества последовательно удаляются из обоих множеств. Если исходные множества эквиваленты, то в конце получим два пустых множества.

\begin{lstlisting}[language=Lisp]
(defun set-equal (set1 set2)
	(cond 
		((null set1) (null set2)) ;3 тест
		((null set2) Nil) ;4 тест
		(t (set-equal (cdr set1) (remove (car set1) set2)))
	)
)

(set-equal '(1 2 3) '(1 2 3)) => T
(set-equal '() '()) => T
(set-equal '(1 2) '(1 2 3)) => NIL
(set-equal '(1 2) '(1)) => NIL
\end{lstlisting}



\clearpage
\section*{3. Напишите свои необходимые функции, которые обрабатывают таблицу из 4-х точечных пар: (страна . столица), и возвращают по стране - столицу, а по столице — страну.}



\begin{lstlisting}[language=Lisp]
; 1. Используя информацию о том, что в таблице ровно 4 точечные пары
(defun find_capital_by_country (table country)
	(cond
		((eql (caar table) country) (cdar table))
		((eql (caadr table) country) (cdadr table))
		((eql (caaddr table) country) (cdaddr table))
		((eql (caaddr (cdr table)) country) (cdaddr (cdr table)))
	)
)

(defun find_country_by_capital (table capital)
	(cond
		((eql (cdar table) capital) (caar table))
		((eql (cdadr table) capital) (caadr table))
		((eql (cdaddr table) capital) (caaddr table))
		((eql (cdaddr (cdr table)) capital) (caaddr (cdr table)))
	)
)

; Используя some
; Функция (SOME TEST LIST1 ... LISTN) выполняет действия предиката TEST над 
; CAR-элементами списков LIST1,...,LISTN, затем - над CADR-обьектами каждого
; списка и т.д. до тех пор, пока тест не вернет значение, отличное от NIL, 
; или не встретится конец списка. Если тест возвращает значение, отличное от NIL, 
; функция SOME возвращает это значение, если же конец списка достигнут, 
; функция SOME возвращает NIL.

(defun find_capital_by_country (table country)
	(some 
		#'(lambda (row) (cond ((eql (car row) country) (cdr row)))) 
		table
	) 
)

(defun find_country_by_capital (table capital)
	(some 
		#'(lambda (row) (cond ((eql (cdr row) capital) (car row)))) 
		table
	) 
)



; Используя assoc/rassoc
; Функция assoc (rassoc) выбирает из ассоциативного списка, заданного вторым аргументом, 
; первую пару, в которой первый (второй) элемент совпадает со значением первого аргумента. 

(defun find_capital_by_country (table country)
	(cdr (assoc country table))
)

(defun find_country_by_capital (table capital)
	(car (rassoc capital table))
)
\end{lstlisting}

Все варианты функций проверялись на следующих тестах:
\begin{lstlisting}[language=Lisp]
(defvar table)
(setq table '((Russia . Moscow) (GreatBritain . London)                  (France . Paris) (Italy . Roma)))

(find_capital_by_country table 'Russia) => MOSCOW
(find_capital_by_country table 'Italy) => ROMA
(find_capital_by_country table 'USA) =>"(NO SUCH COUNTRY)" NIL

(find_country_by_capital table 'Moscow) => Russia
(find_country_by_capital table 'Paris) => FRANCE
(find_country_by_capital table 'Washington) =>"(NO SUCH CAPITAL)" NIL
\end{lstlisting}

\if 0
$
(defvar table)
(setq table '((Russia . Moscow) (GreatBritain . London)                  (France . Paris) (Italy . Roma)))
(find_capital_by_country table 'Russia)
(find_capital_by_country table 'Italy)
(find_capital_by_country table 'USA)
(find_country_by_capital table 'Moscow)
(find_country_by_capital table 'Paris)
(find_country_by_capital table 'Washington)
$
\fi



\clearpage
\section*{5. Напишите функцию swap-two-element, которая переставляет в списке-аргументе два указанных своими порядковыми номерами элемента в этом списке.}

\begin{lstlisting}[language=Lisp]
; элемент n не входит в результат
(defun list_till_n (n lst) 
	(cond 
		((or (null lst) (= n 0)) Nil)
		(t (cons (car lst) (first_n (- n 1) (cdr lst))))
	)
)

; элементы to и from не входят в результат
(defun list_slice_from_to (lst from to)
	(nthcdr (+ from 1) (list_till_n to lst))
)

(defun swap-two-element-inner (lst index1 index2)
	(append 
		(list_till_n index1 lst)
		(cons (nth index2 lst) (list_slice_from_to lst index1 index2))
		(cons (nth index1 lst) (nthcdr (+ index2 1) lst))	
	)
)


(defun swap-two-element (lst index1 index2)
	(cond 
		((and (< index1 index2) (< -1 index1) (< index2 (length lst)))
			(swap-two-element-inner lst index1 index2)
		)
		((and (> index1 index2) (< -1 index2) (< index1 (length lst)))
			(swap-two-element-inner lst index2 index1)
		)
		((= index1 index2) lst)
	)
)


(swap-two-element '(0 1 2 3 4 5 6 7) 2 4) => (0 1 4 3 2 5 6 7)
(swap-two-element '(0 1 2 3 4 5 6 7) 0 7) => (7 1 2 3 4 5 6 0)
(swap-two-element '(0 1 2 3 4 5 6 7) -1 2) => Nil
(swap-two-element '(0 1 2 3 4 5 6 7) 1 8) => Nil
(swap-two-element '(0 1 2 3 4 5 6 7) 5 3) => (0 1 2 5 4 3 6 7)
(swap-two-element '(0 1 2 3 4 5 6 7) 3 3) => (0 1 2 3 4 5 6 7)
\end{lstlisting}

\if 0
$
(swap-two-element '(0 1 2 3 4 5 6 7) 2 4)
(swap-two-element '(0 1 2 3 4 5 6 7) 0 7)
(swap-two-element '(0 1 2 3 4 5 6 7) 5 3)
(swap-two-element '(0 1 2 3 4 5 6 7) 3 3)
(swap-two-element '(0 1 2 3 4 5 6 7) -1 2)
(swap-two-element '(0 1 2 3 4 5 6 7) 1 8)
$
\fi




	\bibliographystyle{utf8gost705u}  % стилевой файл для оформления по ГОСТу
	
	\bibliography{51-biblio}          % имя библиографической базы (bib-файла)

	
\end{document}
