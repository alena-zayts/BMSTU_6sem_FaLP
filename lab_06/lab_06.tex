

\documentclass[12pt]{report}
\usepackage[utf8]{inputenc}
\usepackage[russian]{babel}
\usepackage[14pt]{extsizes}
\usepackage{listings}
\usepackage{graphicx}
\usepackage{amsmath,amsfonts,amssymb,amsthm,mathtools} 
\usepackage{pgfplots}
\usepackage{filecontents}
\usepackage{float}
\usepackage{indentfirst}
\usepackage{eucal}
\usepackage{enumitem}
%s\documentclass[openany]{book}
\frenchspacing

\usepackage{titlesec}
\titleformat{\section}
{\normalsize\bfseries}
{\thesection}
{1em}{}
\titlespacing*{\chapter}{0pt}{-30pt}{8pt}
\titlespacing*{\section}{\parindent}{*4}{*4}
\titlespacing*{\subsection}{\parindent}{*4}{*4}

\usepackage{indentfirst} % Красная строка

\usetikzlibrary{datavisualization}
\usetikzlibrary{datavisualization.formats.functions}

\usepackage{amsmath}

\usepackage{amssymb}

% Для листинга кода:
\lstset{ %
	language=lisp,                 % выбор языка для подсветки (здесь это С)
	texcl=true,
	extendedchars=\true,
	basicstyle=\small\sffamily, % размер и начертание шрифта для подсветки кода
	numbers=left,               % где поставить нумерацию строк (слева\справа)
	numberstyle=\tiny,           % размер шрифта для номеров строк
	stepnumber=1,                   % размер шага между двумя номерами строк
	numbersep=5pt,                % как далеко отстоят номера строк от подсвечиваемого кода
	showspaces=false,            % показывать или нет пробелы специальными отступами
	showstringspaces=false,      % показывать или нет пробелы в строках
	showtabs=false,             % показывать или нет табуляцию в строках
	frame=single,              % рисовать рамку вокруг кода
	tabsize=2,                 % размер табуляции по умолчанию равен 2 пробелам
	captionpos=t,              % позиция заголовка вверху [t] или внизу [b] 
	breaklines=true,           % автоматически переносить строки (да\нет)
	breakatwhitespace=false, % переносить строки только если есть пробел
	escapeinside={\#*}{*)},  % если нужно добавить комментарии в коде
	%inputencoding=utf8x,
	%extendedchars=\true
}



\usepackage[left=2cm,right=2cm, top=2cm,bottom=2cm,bindingoffset=0cm]{geometry}
% Для измененных титулов глав:
\usepackage{titlesec, blindtext, color} % подключаем нужные пакеты
\definecolor{gray75}{gray}{0.75} % определяем цвет
\newcommand{\hsp}{\hspace{20pt}} % длина линии в 20pt
% titleformat определяет стиль
\titleformat{\chapter}[hang]{\Huge\bfseries}{\thechapter\hsp\textcolor{gray75}{|}\hsp}{0pt}{\Huge\bfseries}


% plot
\usepackage{pgfplots}
\usepackage{filecontents}
\usetikzlibrary{datavisualization}
\usetikzlibrary{datavisualization.formats.functions}

\begin{document}
	%\def\chaptername{} % убирает "Глава"
	\thispagestyle{empty}
	\begin{titlepage}
		\noindent \begin{minipage}{0.15\textwidth}
			\includegraphics[width=\linewidth]{img/b_logo}
		\end{minipage}
		\noindent\begin{minipage}{0.9\textwidth}\centering
			\textbf{Министерство науки и высшего образования Российской Федерации}\\
			\textbf{Федеральное государственное бюджетное образовательное учреждение высшего образования}\\
			\textbf{~~~«Московский государственный технический университет имени Н.Э.~Баумана}\\
			\textbf{(национальный исследовательский университет)»}\\
			\textbf{(МГТУ им. Н.Э.~Баумана)}
		\end{minipage}
		
		\noindent\rule{18cm}{3pt}
		\newline\newline
		\noindent ФАКУЛЬТЕТ $\underline{\text{«Информатика и системы управления»}}$ \newline\newline
		\noindent КАФЕДРА $\underline{\text{«Программное обеспечение ЭВМ и информационные технологии»}}$\newline\newline\newline\newline\newline
		
		\begin{center}
			\noindent\begin{minipage}{1.1\textwidth}\centering
				\Large\textbf{  Отчет по лабораторной работе №6}\newline
				\textbf{по дисциплине <<Функциональное и логическое}\newline
				\textbf{~~~программирование>>}\newline\newline
			\end{minipage}
		\end{center}
		
		\noindent\textbf{Тема} $\underline{\text{Использование функционалов}}$\newline\newline
		\noindent\textbf{Студент} $\underline{\text{Зайцева А. А.~~~~~~~~~~~~~~~~~~~~~~~~~~~~~~~~~~~~~~~~~~}}$\newline\newline
		\noindent\textbf{Группа} $\underline{\text{ИУ7-62Б~~~~~~~~~~~~~~~~~~~~~~~~~~~~~~~~~~~~~~~~~~~~~~~~~~}}$\newline\newline
		\noindent\textbf{Оценка (баллы)} $\underline{\text{~~~~~~~~~~~~~~~~~~~~~~~~~~~~~~~~~~~~~~~~~~~~~~~~~}}$\newline\newline
		\noindent\textbf{Преподаватели} $\underline{\text{Толпинская Н.Б., Строганов Ю. В.~~~~~~~~~~~~~~~~~~~~~~~~~~~~}}$\newline\newline\newline
		
		\begin{center}
			\vfill
			Москва~---~\the\year
			~г.
		\end{center}
	\end{titlepage}


\chapter*{Практические задания}	

Используя функционалы:


\section*{1. Напишите функцию, которая уменьшает на 10 все числа из списка-аргумента этой функции.}

1. Если все элементы списка -- числа. 
\begin{lstlisting}[language=Lisp]
; используя функционал mapcar
(defun all_minus_10 (lst)
	(mapcar #'(lambda (x) (- x 10)) lst)
)

; используя функционал mapcan
(defun all_minus_10 (lst)
	(mapcan #'(lambda (x) (cons (- x 10) Nil)) lst)
)

(all_minus_10 '(0 10 -10 5.5 2/3)) => (-10 0 -20 -4.5 -28/3)
(all_minus_10 Nil) => NIL
\end{lstlisting}

2. Элементы списка -- любые объекты.

a) Работа только по верхнему уровню. 

\begin{lstlisting}[language=Lisp]
; используя функционал mapcar.
(defun all_minus_10 (lst)
	(mapcar 
		#'(lambda (x) (cond ((numberp x) (- x 10)) (x)))
		lst
	)
)

; рекурсивно
; а) с помощью  рекурсии, собирающей результат на выходе
(defun all_minus_10 (lst)
	(cond 
		((null lst) Nil)
		((numberp (car lst)) (cons (- (car lst) 10) (all_minus_10 (cdr lst))))
		(t (cons (car lst) (all_minus_10 (cdr lst))))
	)
)	



; б) через "хвостовую рекурсию" 
(defun all_minus_10_inner (lst result)
	(cond 
		((null lst) (reverse result))
		((numberp (car lst)) (all_minus_10_inner (cdr lst) (cons (- (car lst) 10) result)))
		(t (all_minus_10_inner (cdr lst) (cons (car lst) result)))
	)
)

(defun all_minus_10 (lst) 
	(all_minus_10_inner lst ())
)


(all_minus_10 '(0 a "abc" (1 k) 2/3)) => (-10 A "abc" (1 K) -28/3)
\end{lstlisting}

б) Работа по всем уровням структурированного списка.
\begin{lstlisting}[language=Lisp]
; вспомогательная функция, так как mapcar работает только со списками 
; (не с точечными парами).
(defun all_minus_10_cons (cns)
	(cond 
		((and (numberp (car cns)) (numberp (cdr cns))) (cons (- (car cns) 10) (- (cdr cns) 10)))
		((numberp (car cns)) (cons (- (car cns) 10) (cdr cns)))
		((and (numberp (cdr cns)) (atom (car cns))) (cons (car cns) (- (cdr cns) 10)))
		((and (numberp (cdr cns)) (consp (car cns))) (cons (all_minus_10_cons (car cns)) (- (cdr cns) 10)))
		((consp (car cns)) (cons (all_minus_10_cons (car cns)) (cdr cns)))
		(t (cons (car cns) (cdr cns))))
)
; используя функционал mapcar.
; Для определения того, является ли x точечной парой, используется проверка (atom (cdr x)) 
; вместо (consp x) из-за следующих результатов:
; (listp '(k . 10))=(consp '(k . 10)) => T)
; (consp '(1 (K . 10)))=(listp '(1 (K . 10))) => T
(defun all_minus_10 (lst)
	(mapcar 
		#'(lambda (x) 
			(cond 
				((numberp x) (- x 10))
				((atom x) x)
				((atom (cdr x)) (all_minus_10_cons x))
				(t (all_minus_10 x))	
			))
		lst)
)	

; рекурсивно
; а) с помощью  рекурсии, собирающей результат на выходе
(defun all_minus_10 (lst)
	(cond 
		((null lst) Nil)
		((numberp (car lst)) (cons (- (car lst) 10) (all_minus_10 (cdr lst))))
		((atom (car lst)) (cons (car lst) (all_minus_10 (cdr lst))))
		((atom (cdr (car lst))) (cons (all_minus_10_cons (car lst)) (all_minus_10 (cdr lst))))
		(t (cons (all_minus_10 (car lst)) (all_minus_10 (cdr lst))))
	)
)	

; б) через "хвостовую рекурсию" 
(defun all_minus_10_inner (lst result)
	(cond 
		((null lst) (reverse result))
		((numberp (car lst)) (all_minus_10_inner (cdr lst) (cons (- (car lst) 10) result)))
		((atom (car lst)) (all_minus_10_inner (cdr lst) (cons (car lst) result)))
		((atom (cdr (car lst))) (all_minus_10_inner (cdr lst) (cons (all_minus_10_cons (car lst)) result)))
		(t (all_minus_10_inner (cdr lst) (cons (all_minus_10_inner (car lst) ()) result)))
	)
)

(defun all_minus_10 (lst) 
	(all_minus_10_inner lst ())
)


(all_minus_10 '(0 a "abc" (1 (k . 10)) 2/3 ((1 . f) . 10))) => 
	(-10 A "abc" (-9 (K . 0)) -28/3 ((-9 . F) . 0))
\end{lstlisting}



\clearpage
\section*{2. Напишите функцию, которая умножает на заданное число-аргумент все числа из заданного списка-аргумента, когда:}

1. Все элементы списка -- числа.
\begin{lstlisting}[language=Lisp]
; используя функционал mapcar
(defun mult_all (lst num)
	(mapcar #'(lambda (x) (* x num)) lst)
)
(mult_all '(0 10 -10 5.5 2/3) 2) => (0 20 -20 11.0 4/3)
\end{lstlisting}

2. Элементы списка -- любые объекты.

\begin{lstlisting}[language=Lisp]
; используя функционал mapcar.
(defun mult_all (lst num)
	(mapcar 
		#'(lambda (x) (cond ((numberp x) (* x num)) (x)))
		lst
	)
)

; рекурсивно
; а) с помощью  рекурсии, собирающей результат на выходе
(defun mult_all (lst num)
	(cond 
		((null lst) Nil)
		((numberp (car lst)) (cons (* (car lst) num) (mult_all (cdr lst) num)))
		(t (cons (car lst) (mult_all (cdr lst) num)))
	)
)	

; б) через "хвостовую рекурсию" 
(defun mult_all_inner (lst result num)
	(cond 
		((null lst) (reverse result))
		((numberp (car lst)) (mult_all_inner (cdr lst) (cons (* (car lst) num) result) num))
		(t (mult_all_inner (cdr lst) (cons (car lst) result) num))
	)
)

(defun mult_all (lst num) 
	(mult_all_inner lst () num)
)


(mult_all '(0 a "abc" (1 k) 2/3) 2) => (0 A "abc" (1 K) 4/3)
\end{lstlisting}



\section*{3. Написать функцию, которая по своему списку-аргументу lst определяет является ли он палиндромом (то есть равны ли lst и (reverse lst))}

1. Проверка на равенство исходного списка и инвертированного исходного списка.
\begin{lstlisting}[language=Lisp]
(defun is_palindrome (lst) 
	(equalp lst (reverse lst))
)
\end{lstlisting}



2. Проверка на равенство инвертированной первой половины исходного списка и второй половины исходного списка (если список нечетной длины, то центральный элемент не попадает ни в первый, ни во второй список) (этот вариант и было предложено реализовать).

(nthcdr n lst) выполняет для списка lst операцию cdr n раз, и возвращает результат. (floor n) усекает значения по нижней границе. (ceiling n) усекает значения по верхней границе.
\begin{lstlisting}[language=Lisp]
(defun first_n_rev (n lst res) 
	(cond 
		((or (null lst) (= n 0)) res)
		(t (first_n_rev (- n 1) (cdr lst) (cons (car lst) res)))
	)
)

(defun is_palindrome (lst)
	(let ((half_len (/ (length lst) 2)))
		(equalp 
			(first_n_rev (floor half_len) lst ()) 
			(nthcdr (ceiling half_len) lst)
		)
	)
)
\end{lstlisting}



\clearpage
3. Рекурсивно: сравнить первый и последний элемент исходного списка, первый и последний элемент исходного списка без первого и последнего элемента, и так далее (если длина списка нечетная, то центральный элемент ни с чем не сравнивается).

\begin{lstlisting}[language=Lisp]
; в данной задаче допустимо возвращать инвертированный список
(defun list_without_last (lst res)
	(cond
		((null (cdr lst)) res)
		(t (list_without_last (cdr lst) (cons (car lst) res)))
	)
)

(defun is_palindrome (lst)
	(cond 
		((null (cdr lst)) t)
		((eql (car lst) (car (last lst))) ;т. к. (last '(1 2))=>(2)
			(is_palindrome (list_without_last (cdr lst) ())))
	)
)
\end{lstlisting}


Все варианты функций проверялись на следующих тестах:
\begin{lstlisting}[language=Lisp]
(is_palindrome Nil) => T
(is_palindrome '(1)) => T
(is_palindrome '(1 2 3)) => NIL
(is_palindrome '(1 2 1)) => T
(is_palindrome '(1 2 3 1)) => NIL
(is_palindrome '(1 2 2 1)) => T
\end{lstlisting}

\if 0
$
(is_palindrome Nil)
(is_palindrome '(1))
(is_palindrome '(1 2 3))
(is_palindrome '(1 2 1))
(is_palindrome '(1 2 3 1))
(is_palindrome '(1 2 2 1))
$
\fi

\clearpage
\section*{4. Написать предикат set-equal, который возвращает t, если два его множества-аргумента содержат одни и те же элементы, порядок которых не имеет значения}

Все элементы первого множества последовательно удаляются из обоих множеств. Если исходные множества эквиваленты, то в конце получим два пустых множества.

\begin{lstlisting}[language=Lisp]
(defun set-equal (set1 set2)
	(cond 
		((null set1) (null set2)) ;3 тест
		((null set2) Nil) ;4 тест
		(t (set-equal (cdr set1) (remove (car set1) set2)))
	)
)

(set-equal '(1 2 3) '(1 2 3)) => T
(set-equal '() '()) => T
(set-equal '(1 2) '(1 2 3)) => NIL
(set-equal '(1 2) '(1)) => NIL
\end{lstlisting}


\section*{5. Написать функцию которая получает как аргумент список чисел, а возвращает список квадратов этих чисел в том же порядке.}

1. Используя функционал mapcar
\begin{lstlisting}[language=Lisp] 
(defun squares (lst)
	(mapcar #'(lambda (x) (* x x)) lst)
)

(squares '(0 10 -10 5.5 2/3)) => (0 100 100 30.25 4/9)
\end{lstlisting}

2. Рекурсивно

\begin{lstlisting}[language=Lisp]
; а) с помощью  рекурсии, собирающей результат на выходе
(defun squares (lst)
	(cond 
		((null lst) Nil)
		(t (let ((x (car lst))) (cons (* x x) (squares (cdr lst)))))
	)
)	




; б) через "хвостовую рекурсию" 
(defun squares_inner (lst result)
	(cond 
		((null lst) (reverse result))
		(t (squares_inner (cdr lst) (let ((x (car lst))) (cons (* x x) result))))
	)
)

(defun squares (lst) 
	(squares_inner lst ())
)
\end{lstlisting}

\section*{6. Напишите функцию, select-between, которая из списка-аргумента, содержащего только числа, выбирает только те, которые расположены между двумя указанными границами-аргументами и возвращает их в виде списка (упорядоченного по возрастанию списка чисел (+ 2 балла)).}

(Границы не включительно)

1. Используя функционал remove-if-not, сортировка по неубываню после выделения элементов, которые расположены между двумя указанными границами-аргументами.

Сортировка по неубыванию:
\begin{lstlisting}[language=Lisp]
; блочная (карманная, корзинная) сортировка
(defun my_sort (lst)
	(cond ((null lst) Nil)
		(t (nconc 
			(my_sort (remove-if-not (lambda (x) (> (car lst) x)) (cdr lst)))
			(remove-if-not (lambda (x) (= (car lst) x)) lst)
			(my_sort (remove-if-not (lambda (x) (< (car lst) x)) (cdr lst))))
		)
	)
)

; сортировка выбором

; в эту функцию не должен попадать пустой список 
; (лямбда-функция принимает строго 2 аргумента)
(defun list_max (lst)
	(reduce 
		#'(lambda (a b) (cond ((> a b) a) (b)))
		lst)
)

(defun my_sort_inner (lst res)
	(cond 
		((null lst) res)
		(t (let* 
				(
					(cur_max (list_max lst))
					(lst_rest (remove cur_max lst :count 1))
				) 
				(my_sort_inner lst_rest (cons cur_max res))
			)
		)
	)
)

(defun my_sort (lst)
	(my_sort_inner lst ())
)
\end{lstlisting}

Функция
\begin{lstlisting}[language=Lisp]	
(defun select-between-inner (lst a b)
	(remove-if-not #'(lambda (x) (< a x b)) lst)
)

(defun select-between (lst a b)
	(cond 
		((< a b) (select-between-inner lst a b))
		(t (select-between-inner lst b a))
	)
)
(defun select-between-sorted (lst a b)
	(my_sort (select-between lst a b))
)
\end{lstlisting}


2. Отсортировать список по невозрастанию. Оставить часть списка, начиная с первого элемента, меньшего верхней границы. Инвертировать список (получаем отсортированный по неубыванию). Оставить часть списка, начиная с первого элемента, большего нижней границы.

Блочная сортировка по невозрастанию:
\begin{lstlisting}[language=Lisp]
(defun my_sort (lst)
	(cond ((null lst) Nil)
		(t (nconc 
			(my_sort (remove-if-not (lambda (x) (< (car lst) x)) (cdr lst)))
			(remove-if-not (lambda (x) (= (car lst) x)) lst)
			(my_sort (remove-if-not (lambda (x) (> (car lst) x)) (cdr lst))))
		))
)
\end{lstlisting}

Функция
\begin{lstlisting}[language=Lisp]
(defun leave_upper (lst b)
	(member (find-if #'(lambda (x) (< x b)) lst) lst)
)
	
(defun leave_lower (lst a)
	(member (find-if #'(lambda (x) (> x a)) lst) lst)
)

(defun select-between-sorted-inner (lst a b)
	(leave_lower 
		(reverse 
			(leave_upper 
				(my_sort lst) 
				b)
			)
		a)
)

(defun select-between-sorted (lst a b)
	(cond
		((< a b) (select-between-sorted-inner lst a b))
		(t (select-between-sorted-inner lst b a))
	)
)

(select-between-sorted '(1 3 2 4 5 7 6) 1 7) => (2 3 4 5 6)
(select-between-sorted '(1 3 2 4 5 7 6) 6 2) => (3 4 5)
\end{lstlisting}

(Рекурсивно -- см. 7 лабораторная работа, номер 5)

\section*{7. Написать функцию, вычисляющую декартово произведение двух своих списков-аргументов. (Напомним, что А х В это множество всевозможных пар (a b), где а принадлежит А, принадлежит В.)}

1. С помощью функционалов mapcar и mapcan
\begin{lstlisting}[language=Lisp]
(defun decart (lstx lsty) 
	(mapcan 
		#'(lambda (x) (mapcar 
					#'(lambda (y) (cons x (cons y Nil))) 
					lsty
					))
		lstx)
)	
(decart '(a b) '(1 2)) => ((a 1) (a 2) (b 1) (b 2))
	
\end{lstlisting}

\clearpage
2. Рекурсивно
\begin{lstlisting}[language=Lisp]
; по принципу move-to с лекции
; составляет список из каждого элемента lst и cons-ячейки cns
(defun decart_row (lst cns res) 
	(cond
		((null lst) res)
		(t (decart_row (cdr lst) cns (cons (cons (car lst) cns) res)))
	)
)

; по принципу move-to с лекции 
; (nconc вместо cons, lsty неизменен, lstx уменьшается в размере)
(defun decart_inner (lstx lsty res) 
	(cond
		((null lstx) res)
		(t (decart_inner 
				(cdr lstx) 
				lsty 
				(nconc 
					(decart_row lsty (cons (car lstx) Nil) ())
					res
				)
			)
		)
	)
)	

(defun decart (lstx lsty) 
	(decart_inner lstx lsty ())
)

(decart '(a b) '(1 2)) => ((2 B) (1 B) (2 A) (1 A))	
\end{lstlisting}

\clearpage
\section*{8. Почему так реализовано reduce, в чем причина?}

Функционал reduce выполняет следующее преобразование исходного списка L=>(e1 e2 … en) с использованием начального значения A и бинарной операции-функции F: (reduce F L A) = (F(…(F(F A e1) e2))…en). Можно сказать, что значение A логически добавляется в начало списка L.

\begin{itemize}
	\item Если список содержит ровно один элемент и начальное значение не задано, то этот элемент возвращается, а функция не вызывается. 
	\item Если список пуст и задано начальное значение, то возвращается начальное значение, а функция не вызывается. 
	\item Если список пуст и начальное значение не задано, то функция вызывается без аргументов, и reduce возвращает то, что вернет функция. Это единственный случай, когда функция вызывается с другим количеством аргументов, кроме двух.
\end{itemize}

В требованиях сказано, что функция должна принимать в качестве аргументов два элемента последовательности или результаты объединения этих элементов. Функция также должна быть способна не принимать никаких аргументов.

\begin{lstlisting}[language=Lisp]
(reduce #'+ ()) => 0
; в задании указано, что (reduce \#'+ 0) => 0, но
(reduce #'+ 0) => Value of 0 in (REDUCE #'+ 0) is 0, not a SEQUENCE.
; возможно, имелось в виду следующее:
(reduce #'+ () :initial-value 0) => 0
; или
(reduce #'+ '(0)) => 0
\end{lstlisting}

(reduce \#'+ () :initial-value 0): список пуст, начальное значение указано, следовательно, функция + не вызывается, и reduce возвращает начальное значение.

(reduce \#'+ '(0)): список состоит из одного элемента, начальное значение не указано, следовательно, функция + не вызывается, и reduce возвращает единственный элемент списка.

(reduce \#'+ ()): список пуст, начальное значение не указано, следовательно, функция + вызывается без аргументов, и reduce возвращает то, что вернет функция +. (+) => 0, поэтому и (reduce \#'+ ()) => 0.

Примечание: (reduce \#'* ()) => по той же причине, но для функций - и / такой вызов осуществить невозможно, так как они не могут принимать ноль аргументов.

\clearpage
\section*{9. Пусть list-of-list список, состоящий из списков. Написать функцию, которая вычисляет сумму длин всех элементов list-of-list, т.е. например для аргумента ((1 2) (3 4)) -> 4}


\begin{lstlisting}[language=Lisp]
; с использованием функционалов mapcar и reduce 
(defun sum_of_legths (ll) 
	(reduce #'+ (mapcar #'length ll))
)

; рекурсивно
(defun sum_of_legths_inner (ll sum) 
	(cond 
		((null ll) sum)
		(t (sum_of_legths_inner (cdr ll) (+ sum (length (car ll)))))
	)
)

(defun sum_of_legths (ll) 
	(sum_of_legths_inner ll 0)
)

(sum_of_legths '()) => 0
(sum_of_legths '(())) => 0
(sum_of_legths '((1 2))) => 2
(sum_of_legths '((1 2) (3 4))) => 4
(sum_of_legths '((1 2) (3) (4 5))) => 5
\end{lstlisting}

\if 0
$
(sum_of_legths '())
(sum_of_legths '(()))
(sum_of_legths '((1 2)))
(sum_of_legths '((1 2) (3 4)))
(sum_of_legths '((1 2) (3) (4 5)))
$
\fi

	%\bibliographystyle{utf8gost705u}  % стилевой файл для оформления по ГОСТу
	
	%\bibliography{51-biblio}          % имя библиографической базы (bib-файла)

	
\end{document}
