\documentclass[12pt]{report}
\usepackage[utf8]{inputenc}
\usepackage[russian]{babel}
\usepackage[14pt]{extsizes}
\usepackage{listings}
\usepackage{graphicx}
\usepackage{amsmath,amsfonts,amssymb,amsthm,mathtools} 
\usepackage{pgfplots}
\usepackage{filecontents}
\usepackage{float}
\usepackage{indentfirst}
\usepackage{eucal}
\usepackage{enumitem}
%s\documentclass[openany]{book}
\frenchspacing

\usepackage{titlesec}
\titleformat{\section}
{\normalsize\bfseries}
{\thesection}
{1em}{}
\titlespacing*{\chapter}{0pt}{-30pt}{8pt}
\titlespacing*{\section}{\parindent}{*4}{*4}
\titlespacing*{\subsection}{\parindent}{*4}{*4}

\usepackage{indentfirst} % Красная строка

\usetikzlibrary{datavisualization}
\usetikzlibrary{datavisualization.formats.functions}

\usepackage{amsmath}

\usepackage{amssymb}

% Для листинга кода:
\lstset{ %
	language=lisp,                 % выбор языка для подсветки (здесь это С)
	texcl=true,
	extendedchars=\true,
	basicstyle=\small\sffamily, % размер и начертание шрифта для подсветки кода
	numbers=left,               % где поставить нумерацию строк (слева\справа)
	numberstyle=\tiny,           % размер шрифта для номеров строк
	stepnumber=1,                   % размер шага между двумя номерами строк
	numbersep=5pt,                % как далеко отстоят номера строк от подсвечиваемого кода
	showspaces=false,            % показывать или нет пробелы специальными отступами
	showstringspaces=false,      % показывать или нет пробелы в строках
	showtabs=false,             % показывать или нет табуляцию в строках
	frame=single,              % рисовать рамку вокруг кода
	tabsize=2,                 % размер табуляции по умолчанию равен 2 пробелам
	captionpos=t,              % позиция заголовка вверху [t] или внизу [b] 
	breaklines=true,           % автоматически переносить строки (да\нет)
	breakatwhitespace=false, % переносить строки только если есть пробел
	escapeinside={\#*}{*)},  % если нужно добавить комментарии в коде
	%inputencoding=utf8x,
	%extendedchars=\true
}



\usepackage[left=2cm,right=2cm, top=2cm,bottom=2cm,bindingoffset=0cm]{geometry}
% Для измененных титулов глав:
\usepackage{titlesec, blindtext, color} % подключаем нужные пакеты
\definecolor{gray75}{gray}{0.75} % определяем цвет
\newcommand{\hsp}{\hspace{20pt}} % длина линии в 20pt
% titleformat определяет стиль
\titleformat{\chapter}[hang]{\Huge\bfseries}{\thechapter\hsp\textcolor{gray75}{|}\hsp}{0pt}{\Huge\bfseries}


% plot
\usepackage{pgfplots}
\usepackage{filecontents}
\usetikzlibrary{datavisualization}
\usetikzlibrary{datavisualization.formats.functions}

\begin{document}
	%\def\chaptername{} % убирает "Глава"
	\thispagestyle{empty}
	\begin{titlepage}
		\noindent \begin{minipage}{0.15\textwidth}
			\includegraphics[width=\linewidth]{img/b_logo}
		\end{minipage}
		\noindent\begin{minipage}{0.9\textwidth}\centering
			\textbf{Министерство науки и высшего образования Российской Федерации}\\
			\textbf{Федеральное государственное бюджетное образовательное учреждение высшего образования}\\
			\textbf{~~~«Московский государственный технический университет имени Н.Э.~Баумана}\\
			\textbf{(национальный исследовательский университет)»}\\
			\textbf{(МГТУ им. Н.Э.~Баумана)}
		\end{minipage}
		
		\noindent\rule{18cm}{3pt}
		\newline\newline
		\noindent ФАКУЛЬТЕТ $\underline{\text{«Информатика и системы управления»}}$ \newline\newline
		\noindent КАФЕДРА $\underline{\text{«Программное обеспечение ЭВМ и информационные технологии»}}$\newline\newline\newline\newline\newline
		
		\begin{center}
			\noindent\begin{minipage}{1.1\textwidth}\centering
				\Large\textbf{  Отчет по лабораторной работе №7}\newline
				\textbf{по дисциплине <<Функциональное и логическое}\newline
				\textbf{~~~программирование>>}\newline\newline
			\end{minipage}
		\end{center}
		
		\noindent\textbf{Тема} $\underline{\text{Рекурсивные функции}}$\newline\newline
		\noindent\textbf{Студент} $\underline{\text{Зайцева А. А.~~~~~~~~~~~~~~~~~~~~~~~~~~~~~~~~~~~~~~~~~~}}$\newline\newline
		\noindent\textbf{Группа} $\underline{\text{ИУ7-62Б~~~~~~~~~~~~~~~~~~~~~~~~~~~~~~~~~~~~~~~~~~~~~~~~~~}}$\newline\newline
		\noindent\textbf{Оценка (баллы)} $\underline{\text{~~~~~~~~~~~~~~~~~~~~~~~~~~~~~~~~~~~~~~~~~~~~~~~~~}}$\newline\newline
		\noindent\textbf{Преподаватели} $\underline{\text{Толпинская Н.Б., Строганов Ю. В.~~~~~~~~~~~~~~~~~~~~~~~~~~~~}}$\newline\newline\newline
		
		\begin{center}
			\vfill
			Москва~---~\the\year
			~г.
		\end{center}
	\end{titlepage}


\chapter*{Практические задания}	

Используя рекурсию:


\section*{1. Написать хвостовую рекурсивную функцию my-reverse, которая развернет верхний уровень своего списка-аргумента lst.}

\begin{lstlisting}[language=Lisp]
(defun move-to (lst result)
	(cond 
		((null lst) result)
		(t (move-to (cdr lst) (cons (car lst) result)))
	)
)

(defun my-reverse (lst)
	(move-to lst ())
)

(my-reverse '(1 a ((2 . 3) 5 6) 7 8)) => (8 7 ((2 . 3) 5 6) A 1)
\end{lstlisting}



\section*{2. Написать функцию, которая возвращает первый элемент списка-аргумента, который сам является непустым списком.}

1. Хвостовая рекурсия
\begin{lstlisting}[language=Lisp]
; дополнительная проверка (listp (cdr x)) нужна из-за следующего результата:
;(listp '(1 . 2)) => T
	
(defun f (lst)
	(cond
		((null lst) Nil)
		(
			((lambda (x) (and (listp x) (listp (cdr x)) (not (null x)))) 
				(car lst))
			(car lst)
		)
		(t (f (cdr lst)))
	)
)
(f '(0 () (1 . 2) (2 3) (3) a)) => (2 3)
(f '(1)) => Nil
\end{lstlisting}

\clearpage
2. С помощью функционала find-if
\begin{lstlisting}[language=Lisp]
(defun f (lst)
	(find-if 
	 #'(lambda (x) 
	 	(and (listp x) (listp (cdr x)) (not (null x)))
	 	)
	 lst
	 )
)
(f '(0 () (1 . 2) (2 3) (3) a)) => (2 3)
(f '(1)) => Nil
\end{lstlisting}


\section*{3. Написать функцию, которая выбирает из заданного списка только числа между двумя заданными границами.}

1. Хвостовая рекурсия
\begin{lstlisting}[language=Lisp]
; а) один cond, но рекурсивный вызов встречается в теле дважды
(defun move_to (lst res a b)
	(cond
		((null lst) (reverse res))
		((and (numberp (car lst)) (< a (car lst) b))
			(move_to (cdr lst) (cons (car lst) res) a b))
		((move_to (cdr lst) res a b))
	)
)
; б) два cond, но рекурсивный вызов встречается в теле единожды
(defun move_to (lst res a b)
	(cond
		((null lst) (reverse res))
		(t (move_to (cdr lst) 
			(cond 
				((and (numberp (car lst)) (< a (car lst) b))
					(cons (car lst) res))
				(res)
			) a b)
		 )
	 )
)
(defun select_between (lst a b)
	(cond
		((< a b) (move_to lst () a b))
		(t (move_to lst () b a))
	)
)
\end{lstlisting}

2. Рекурсия, которая собирает результат на выходе.
\begin{lstlisting}[language=Lisp]
(defun select_between (lst a b)
	(cond
		((null lst) Nil)
		((and (numberp (car lst)) (< a (car lst) b)) (cons (car lst) (select_between (cdr lst) a b)))
		(t (select_between (cdr lst) a b))
	)
)
(select_between '(0 3 7 a 5 (4 3) 1 6) 1 6) => (3 5)
\end{lstlisting}


\section*{4. Напишите рекурсивную функцию, которая умножает на заданное число-аргумент все числа из заданного списка-аргумента, когда:}

1. Все элементы списка -- числа.
\begin{lstlisting}[language=Lisp]
; а) с помощью  рекурсии, собирающей результат на выходе
(defun mult_all (lst num)
	(cond
		((null lst) Nil)
		(t (cons (* (car lst) num) (mult_all (cdr lst) num)))
	)
)

; б) через хвостовую рекурсию
(defun move_to (lst res num)
	(cond
		((null lst) (reverse res))
		(t (move_to (cdr lst) (cons (* (car lst) num) res) num))
	)
)

(defun mult_all (lst num)
	(move_to lst () num)
)

(mult_all '(0 10 -10 5.5 2/3) 2) => (0 20 -20 11.0 4/3)
\end{lstlisting}

\clearpage
2. Элементы списка -- любые объекты. 

C помощью  рекурсии, собирающей результат на выходе. 

a) Работа только по верхнему уровню.
\begin{lstlisting}[language=Lisp]
(defun mult_all (lst num)
	(cond 
		((null lst) Nil)
		((numberp (car lst)) (cons (* (car lst) num) (mult_all (cdr lst) num)))
		(t (cons (car lst) (mult_all (cdr lst) num)))
	)
)	
(mult_all '(0 a "abc" (1 k) 2/3 ((1 . 2) . 3)) 2) => (0 A "abc" (1 K) 4/3 ((1 . 2) . 3))
\end{lstlisting}

б) Работа по всем уровням структурированного списка.
\begin{lstlisting}[language=Lisp]
; Для определения того, является ли x точечной парой, используется проверка (atom (cdr x)) 
; вместо (consp x) из-за следующего результата:
; (listp '(k . 10))=(consp '(k . 10)) => T

; вспомогательная функция, для точечных пар
(defun mult_all_cons (cns num)
	(cond 
		((and (numberp (car cns)) (numberp (cdr cns))) (cons (* (car cns) num) (* (cdr cns) num)))
		((numberp (car cns)) (cons (* (car cns) num) (cdr cns)))
		((and (numberp (cdr cns)) (atom (car cns))) (cons (car cns) (* (cdr cns) num)))
		((and (numberp (cdr cns)) (consp (car cns))) (cons (mult_all_cons (car cns) num) (* (cdr cns) num)))
		((consp (car cns)) (cons (mult_all_cons (car cns) num) (cdr cns)))
		(t (cons (car cns) (cdr cns)))
	)
)
(defun mult_all (lst num)
	(cond 
		((null lst) Nil)
		((numberp (car lst)) (cons (* (car lst) num) (mult_all (cdr lst) num)))
		((atom (car lst)) (cons (car lst) (mult_all (cdr lst) num)))
		((atom (cdr (car lst))) (cons (mult_all_cons (car lst) num) (mult_all (cdr lst) num)))
		(t (cons (mult_all (car lst) num) (mult_all (cdr lst) num)))
	)
)		
(mult_all '(0 a "abc" (1 k) 2/3 ((1 . 2) . 3)) 2) => (0 A "abc" (2 K) 4/3 ((2 . 4) . 6))
\end{lstlisting}

\section*{5. Напишите функцию select-between, которая из списка-аргумента, содержащего только числа, выбирает только те, которые расположены между двумя указанными границами-аргументами и возвращает их в виде списка (упорядоченного по возрастанию списка чисел (+ 2 балла)).
}

(Границы не включительно)

Сортировка по невозрастанию:
\begin{lstlisting}[language=Lisp]
; блочная (карманная, корзинная) сортировка
(defun my_sort (lst)
	(cond ((null lst) Nil)
		(t (nconc 
			(my_sort (remove-if-not (lambda (x) (< (car lst) x)) (cdr lst)))
			(remove-if-not (lambda (x) (= (car lst) x)) lst)
			(my_sort (remove-if-not (lambda (x) (> (car lst) x)) (cdr lst)))
			)
		)
	)
)
; сортировка выбором

; в эту функцию не должен попадать пустой список 
; (лямбда-функция принимает строго 2 аргумента)
(defun list_min (lst)
	(reduce 
		#'(lambda (a b) (cond ((< a b) a) (b)))
		lst
	)
)
(defun my_sort_inner (lst res)
	(cond 
		((null lst) res)
		(t (let* 
				(
					(cur_min (list_min lst))
					(lst_rest (remove cur_min lst :count 1))
				) 
				(my_sort_inner lst_rest (cons cur_min res))
			)
		)
	)
)

(defun my_sort (lst)
	(my_sort_inner lst ()))
\end{lstlisting}

Сначала исходный список сортируется по невозрастанию. Затем рекурсивно отсекается его голова, пока голова не станет меньше верхней границы. Наконец, отсекается его конец, начиная с элемента, который меньше или равен нижней границе, в процессе чего список инвертируется и становится отсортированным по неубыванию. Такой подход позволяет избежать рекурсии, в которой результат собирается на выходе.

\begin{lstlisting}[language=Lisp]
; в функциях cut\_upper и cut\_lower предполагается, что переданный список 
; отсортирован по невозрастанию
(defun cut_upper (lst b)
	(cond 
		((null lst) Nil)
		((< (car lst) b) lst)
		(t (cut_upper (cdr lst) b))
	)
)
(defun cut_lower (lst a res)
	(cond 
		((null lst) res)
		((> a (car lst)) res)
		(t (cut_lower (cdr lst) a (cons (car lst) res)))
	)
)
(defun select-between-sorted (lst a b)	
	(cut_lower (cut_upper (my_sort lst) b) a ())	
)

(select-between-sorted '(0 3 7 6 5 4 1) 1 6) => (3 4 5)
\end{lstlisting}

\section*{6. Написать рекурсивную версию (с именем rec-add) вычисления суммы чисел заданного списка:}

а) одноуровневого смешанного

\begin{lstlisting}[language=Lisp]
(defun rec-add-inner (lst sum)
	(cond 
		((null lst) sum)
		((numberp (car lst)) (rec-add-inner (cdr lst) (+ sum (car lst))))
		(t (rec-add-inner (cdr lst) sum))
	)
)
(defun rec-add (lst)
	(rec-add-inner lst 0)
)
(rec-add '(1 a 2 -1/2 "abc")) => 5/2
\end{lstlisting}

б) структурированного (то есть элементами могут быть списки)

\begin{lstlisting}[language=Lisp]
(defun rec-add-inner (lst sum)
	(cond 
		((null lst) sum)
		((numberp (car lst)) (rec-add-inner (cdr lst) (+ sum (car lst))))
		((atom (car lst)) (rec-add-inner (cdr lst) sum))
		(t (rec-add-inner (cdr lst) (+ sum (rec-add (car lst)))))
	)
)

(defun rec-add (lst)
	(rec-add-inner lst 0)
)

(rec-add '(1 (1 (1 1)) a 1 ((1 "abc") 1))) => 7
\end{lstlisting}

%(взаимная рекурсия, но по сути -- рекурсия более высокого порядка)


\section*{7. Написать рекурсивную версию с именем recnth функции nth.}

\begin{lstlisting}[language=Lisp]
(defun recnth (n lst)
	(cond
		((null lst) Nil)
		((= n 0) (car lst))
		(t (recnth (- n 1) (cdr lst)))
	)
)

(recnth 0 '()) => Nil
(recnth 2 '(0 1)) => Nil
(recnth 0 '(0 1 2)) => 0
(recnth 2 '(0 (1 1) (2))) => (2)
\end{lstlisting}



\clearpage
\section*{8. Написать рекурсивную функцию allodd, которая возвращает t когда все элементы списка нечетные.}


\begin{lstlisting}[language=Lisp]
; если гарантируется, что все элементы списка -- целые числа
(defun allodd (lst)
	(cond
		((null lst) t)
		((oddp (car lst)) (allodd (cdr lst)))
	)
)

(allodd '(1 3 5)) => T
(allodd '()) => T
(allodd '(2 3)) => Nil
(allodd '(1 3 4)) => Nil

; если такой гарантии нет (то есть если встречается элемент списка,
; который не является целым нечетным числом, возвращается Nil)
(defun allodd (lst)
	(cond
		((null lst) t)
		((and (integerp (car lst)) (oddp (car lst))) (allodd (cdr lst)))
	)
)

(allodd '(1 3 5)) => T
(allodd '(1 3.0)) => Nil
(allodd '(1 a)) => Nil
\end{lstlisting}

\section*{9. Написать рекурсивную функцию, которая возвращает первое нечетное число из списка (структурированного), возможно создавая некоторые вспомогательные функции.}


\begin{lstlisting}[language=Lisp]
; x может быть как атомом, так и структурой
(defun first_odd (x)
	(cond
		((null x) Nil)
		((and (integerp x) (oddp x)) x)
		((atom x) Nil)
		((or (first_odd (car x)) (first_odd (cdr x))))	
	)
)


(first_odd '()) => NIL
(first_odd '(() 2 1)) => 1
(first_odd '(1.0 1)) => 1
(first_odd '(a 1)) => 1
(first_odd '("abc" 1)) => 1
(first_odd '((2 4) 1)) => 1
(first_odd '((2 . 4) 1)) => 1
(first_odd '(((2 . 1) 4) 3)) => 1
(first_odd '((2 4 (2 1)) 5)) => 1
(first_odd '((2 4 (2)) 6 (1 (3 5)))) => 1
\end{lstlisting}

\if 0
$
(first_odd '())
(first_odd '(() 2 1))
(first_odd '(1.0 1))
(first_odd '(a 1))
(first_odd '("abc" 1))
(first_odd '((2 4) 1))
(first_odd '((2 . 4) 1))
(first_odd '(((2 . 1) 4) 3))
(first_odd '((2 4 (2 1)) 5))
(first_odd '((2 4 (2)) 6 (1 (3 5))))
$
\fi


\section*{10. Используя cons-дополняемую рекурсию с одним тестом завершения, написать функцию которая получает как аргумент список чисел, а возвращает список квадратов этих чисел в том же порядке}


\begin{lstlisting}[language=Lisp]
; а) с помощью  рекурсии, собирающей результат на выходе
(defun squares (lst)
	(cond 
		((null lst) Nil)
		(t (cons (* (car lst) (car lst)) (squares (cdr lst))))
	)
)	

; б) через "хвостовую рекурсию" 
(defun squares_inner (lst result)
	(cond 
		((null lst) (reverse result))
		(t (squares_inner (cdr lst) (cons (* (car lst) (car lst)) result)))
	)
)

(defun squares (lst) 
	(squares_inner lst ())
)

(squares '(0 10 -10 5.5 2/3)) => (0 100 100 30.25 4/9)
\end{lstlisting}




	%\bibliographystyle{utf8gost705u}  % стилевой файл для оформления по ГОСТу
	
	%\bibliography{51-biblio}          % имя библиографической базы (bib-файла)

	
\end{document}
