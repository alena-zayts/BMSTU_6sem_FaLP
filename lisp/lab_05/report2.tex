

\documentclass[12pt]{report}
\usepackage[utf8]{inputenc}
\usepackage[russian]{babel}
\usepackage[14pt]{extsizes}
\usepackage{listings}
\usepackage{graphicx}
\usepackage{amsmath,amsfonts,amssymb,amsthm,mathtools} 
\usepackage{pgfplots}
\usepackage{filecontents}
\usepackage{float}
\usepackage{indentfirst}
\usepackage{eucal}
\usepackage{enumitem}
%s\documentclass[openany]{book}
\frenchspacing

\usepackage{titlesec}
\titleformat{\section}
{\normalsize\bfseries}
{\thesection}
{1em}{}
\titlespacing*{\chapter}{0pt}{-30pt}{8pt}
\titlespacing*{\section}{\parindent}{*4}{*4}
\titlespacing*{\subsection}{\parindent}{*4}{*4}

\usepackage{indentfirst} % Красная строка

\usetikzlibrary{datavisualization}
\usetikzlibrary{datavisualization.formats.functions}

\usepackage{amsmath}

\usepackage{amssymb}

% Для листинга кода:
\lstset{ %
	language=lisp,                 % выбор языка для подсветки (здесь это С)
	texcl=true,
	extendedchars=\true,
	basicstyle=\small\sffamily, % размер и начертание шрифта для подсветки кода
	numbers=left,               % где поставить нумерацию строк (слева\справа)
	numberstyle=\tiny,           % размер шрифта для номеров строк
	stepnumber=1,                   % размер шага между двумя номерами строк
	numbersep=5pt,                % как далеко отстоят номера строк от подсвечиваемого кода
	showspaces=false,            % показывать или нет пробелы специальными отступами
	showstringspaces=false,      % показывать или нет пробелы в строках
	showtabs=false,             % показывать или нет табуляцию в строках
	frame=single,              % рисовать рамку вокруг кода
	tabsize=2,                 % размер табуляции по умолчанию равен 2 пробелам
	captionpos=t,              % позиция заголовка вверху [t] или внизу [b] 
	breaklines=true,           % автоматически переносить строки (да\нет)
	breakatwhitespace=false, % переносить строки только если есть пробел
	escapeinside={\#*}{*)},  % если нужно добавить комментарии в коде
	%inputencoding=utf8x,
	%extendedchars=\true
}



\usepackage[left=2cm,right=2cm, top=2cm,bottom=2cm,bindingoffset=0cm]{geometry}
% Для измененных титулов глав:
\usepackage{titlesec, blindtext, color} % подключаем нужные пакеты
\definecolor{gray75}{gray}{0.75} % определяем цвет
\newcommand{\hsp}{\hspace{20pt}} % длина линии в 20pt
% titleformat определяет стиль
\titleformat{\chapter}[hang]{\Huge\bfseries}{\thechapter\hsp\textcolor{gray75}{|}\hsp}{0pt}{\Huge\bfseries}


% plot
\usepackage{pgfplots}
\usepackage{filecontents}
\usetikzlibrary{datavisualization}
\usetikzlibrary{datavisualization.formats.functions}

\begin{document}
	%\def\chaptername{} % убирает "Глава"
	\thispagestyle{empty}
	\begin{titlepage}
		\noindent \begin{minipage}{0.15\textwidth}
			\includegraphics[width=\linewidth]{img/b_logo}
		\end{minipage}
		\noindent\begin{minipage}{0.9\textwidth}\centering
			\textbf{Министерство науки и высшего образования Российской Федерации}\\
			\textbf{Федеральное государственное бюджетное образовательное учреждение высшего образования}\\
			\textbf{~~~«Московский государственный технический университет имени Н.Э.~Баумана}\\
			\textbf{(национальный исследовательский университет)»}\\
			\textbf{(МГТУ им. Н.Э.~Баумана)}
		\end{minipage}
		
		\noindent\rule{18cm}{3pt}
		\newline\newline
		\noindent ФАКУЛЬТЕТ $\underline{\text{«Информатика и системы управления»}}$ \newline\newline
		\noindent КАФЕДРА $\underline{\text{«Программное обеспечение ЭВМ и информационные технологии»}}$\newline\newline\newline\newline\newline
		
		\begin{center}
			\noindent\begin{minipage}{1.1\textwidth}\centering
				\Large\textbf{  Отчет по лабораторной работе №5}\newline
				\textbf{по дисциплине <<Функциональное и логическое}\newline
				\textbf{~~~программирование>>}\newline\newline
			\end{minipage}
		\end{center}
		
		\noindent\textbf{Тема} $\underline{\text{Использование управляющих структур, работа со списками}}$\newline\newline
		\noindent\textbf{Студент} $\underline{\text{Зайцева А. А.~~~~~~~~~~~~~~~~~~~~~~~~~~~~~~~~~~~~~~~~~~}}$\newline\newline
		\noindent\textbf{Группа} $\underline{\text{ИУ7-62Б~~~~~~~~~~~~~~~~~~~~~~~~~~~~~~~~~~~~~~~~~~~~~~~~~~}}$\newline\newline
		\noindent\textbf{Оценка (баллы)} $\underline{\text{~~~~~~~~~~~~~~~~~~~~~~~~~~~~~~~~~~~~~~~~~~~~~~~~~}}$\newline\newline
		\noindent\textbf{Преподаватели} $\underline{\text{Толпинская Н.Б., Строганов Ю. В.~~~~~~~~~~~~~~~~~~~~~~~~~~~~}}$\newline\newline\newline
		
		\begin{center}
			\vfill
			Москва~---~\the\year
			~г.
		\end{center}
	\end{titlepage}
	
\chapter*{Теоретические вопросы}

\section*{1. Структуроразрушающие и не разрушающие структуру списка функции}

Функции, работающие со списками, делятся на:
\begin{itemize}
	\item функции, не разрушающие структуру списка (сохраняется возможность работать с исходным списком);
	\item функции, разрушающие структуру списка (не сохраняется возможность работы с исходным списком, зато функция выполнятся быстрее по сравнению со своим не разрушающим аналогом, так как не создаются копии cons-ячеек).
\end{itemize}

Из-за такого разделения существует много дублирующих функций: функциям, не разрушающим структуру списка (reverse, substitute, ...) соответствуют структуроразрушающие функции, которые, как правило, начинаются с буквы <<n>>, как признак того, что не создаются копии (nreverse, nsubstitute, ...). conc является структуроразрушающим аналогом append, delete -- структуроразрушающим аналогом remove.

\section*{2. Отличие в работе функций cons, list, append, nconс и в их результате}

cons принимает 2 указателя на любые S-выражения и возвращает новую cons-ячейку (списковую ячейку), содержащую 2 значения. Если второе значение не NIL и не другая cons-ячейка, то ячейка печатается как два значения в скобках, разделённые точкой (так называемая точечная пара). Иначе, по сути, эта функция включает значение первого аргумента в начало списка, являющегося значением второго аргумента. 

Функция list, составляющая список из значений своих аргументов (у которого голова -- это первый аргумент, хвост -- все остальные аргументы), создает столько списковых ячеек, сколько аргументов ей было передано. Эта функция относится к особым, поскольку у неё может быть произвольное число аргументов, но при этом все аргументы вычисляются.

append принимает произвольное количество аргументов-списков и соединяет (сливает)  элементы верхнего уровня всех списков в один список. Действие append иногда называют конкатенацией списков. В результате должен быть построен новый список.

Например: (append (list 1 2) (list 3 4)) ==> (1 2 3 4). 

С точки зрения функционального подхода, задача функции append - вернуть список (1 2 3 4) не изменяя ни одну из cons-ячеек в списках-аргументах (1 2) и (3 4). append на самом деле создаёт только две новые cons-ячейки, чтобы хранить значения 1 и 2, соединяя их вместе и делая ссылку из CDR второй ячейки на первый элемент последнего аргумента - списка (3 4). После этого функция возвращает cons-ячейку содержащую 1. Ни одна из входных cons-ячеек не была изменена, и результатом, как и требовалось, является список (1 2 3 4). Единственная хитрость в том, что результат, возвращаемый функцией append имеет общие cons-ячейки со списком (3 4). Таким образом, если последний переданный список будет модифицирован, то  итоговый список будет также изменен.

nconc -- это структуроразрушающая версия append. Как и append, nconc возвращает соединение своих аргументов, но строит такой результат следующим образом: для каждого непустого аргумента-списка, nconc устанавливает в cdr его последней cons-ячейки ссылку на первую cons-ячейку следующего непустого аргумента-списка. После этого она возвращает первый список, который теперь является головой результата-соединения.

Итак, отличия: cons является базисной, list, append, nconc -- нет; list, append, nconc принимают произвольное количество аргументов (причем аргументами append и nconc могут быть только списки), cons -- фиксированное (два); cons создает точечную пару или список (в зависимости от второго аргумента), list, append и nconc -- список; cons и list создают новые списковые ячейки (все), append имеет общие списковые ячейки с последним списком, nconc не создает cons-ячеек; conc является структуроразрушающей, а cons, list и append -- нет.

Пусть (setf lst1 '( a b)); (setf lst2 '(c d).



\begin{lstlisting}[language=Lisp]	
	(cons lst1 lst2) => ((A B) C D)
	lst1 => (A B)
	(list lst1 lst2) => ((A B) (C D))
	lst1 => (A B)
	(append lst1 lst2) => (A B C D)
	lst1 => (A B)
	(nconc lst1 lst2) => (A B C D)
	lst1 => (A B C D)
\end{lstlisting}



	
\chapter*{Практические задания}	

\section*{1. Написать функцию, которая по своему списку-аргументу lst определяет, является ли он палиндромом (то есть равны ли lst и (reverse lst)).}

а) При работе только с верхним уровнем, то есть список '((1 2) 3 (1 2))) считается палиндромом, а список '((1 2) 3 (2 1))) -- нет.

\begin{lstlisting}[language=Lisp]	
; c использованием стандартных функций
(defun is_palindrome (lst) (equalp lst (reverse lst)))

; с использованиемм самостоятельно реализованных функций
(defun my_length (lst &optional (len 0)) 
	(cond 
		((null lst) len)
		((my_length (cdr lst) (+ len 1)))
	)
)

(defun my_equalp (x1 x2)
	(cond 
		((numberp x1) (and (numberp x2) (= x1 x2)))
		((atom x1) (and (atom x2) (eq x1 x2)))
		((consp x1) (and (consp x2) (my_equalp (car x1)  (car x2))         (my_equalp (cdr x1) (cdr x2))))
		((listp x1) (and (listp x2) (= (my_length x1) (my_length x2))      (my_equalp (car x1)  (car x2)) (my_equalp (cdr x1) (cdr x2))))	
	)
)


(defun my_rev_upper (lst) 
	(if (null lst)
		Nil 
		(append 
			(my_rev_upper (cdr lst)) 
			(cons (car lst) Nil))
	)
)

(defun is_palindrome (lst) (my_equalp lst (my_rev_upper lst)))




;тесты
(is_palindrome '(1 2 3)) => NIL
(is_palindrome '(1 2 1)) => T
(is_palindrome '((1) 2 1)) => NIL

(is_palindrome '((1 2) 3 (1 2))) => T
(is_palindrome '((1 2) 3 (2 1))) => NIL
\end{lstlisting}


б) При работе по всем уровням, то есть список '((1 2) 3 (1 2))) не считается палиндромом, а список '((1 2) 3 (2 1))) -- да.

\begin{lstlisting}[language=Lisp]
; с использованиемм самостоятельно реализованных функций	
(defun my_rev_deeper (lst) 
	(if (null lst) 
		Nil 
		(append 
			(my_rev_deeper (cdr lst))
			(if (atom (car lst)) 	
				(cons (car lst) Nil)
				(list (my_rev_deeper (car lst)))
			)
		)
	)
)
	
(defun is_palindrome (lst) (my_equalp lst (my_rev_deeper lst)))
	
;тесты
(is_palindrome '(1 2 3)) => NIL
(is_palindrome '(1 2 1)) => T
(is_palindrome '((1) 2 1)) => NIL

(is_palindrome '((1 2) 3 (1 2))) => Nil
(is_palindrome '((1 2) 3 (2 1))) => T
(is_palindrome '((1 (2 3)) 4 ((3 2) 1))) => T
\end{lstlisting}

\clearpage
\section*{2. Написать предикат set-equal, который возвращает t, если два его множества-аргумента содержат одни и те же элементы, порядок которых не имеет значения}

a) Если элементами множеств являются только атомы.

\begin{lstlisting}[language=Lisp]	
; с использованием встроенных функций

(defun set-equal (set1 set2)
	(if (null set1)  
		(null set2)
		(if (member (car set1) set2) 
			(set-equal (cdr set1) (remove (car set1) set2 :count 1))
			Nil
		)
	)
)
\end{lstlisting}

Комментарии к решению:
\begin{itemize}
	\item дополнительный параметр :count 1 в функции remove указан в учебных целях. Так как множество -- это неупорядоченная совокупность неповторяющихся элементов, то этот параметр можно не указывать;
	\item ветвь if (null set1) необходима, так как если множества совпадают, то при последнем вызове set-equal аргументами будут два пустых списка. Тогда далее будет произведена проверка, является ли (car set1)=Nil членом set2=Nil, результатом которой будет Nil, и задача будет решена неверно.
\end{itemize}


б) Если элементами множеств могут быть любые S-выражения.

\begin{lstlisting}[language=Lisp]	
; с использованиемм самостоятельно реализованных функций

(defun my_member (elem lst)
	(if (null lst)
		Nil
		(if (my_equalp elem (car lst))
			lst
			(my_member elem (cdr lst)))
	)
)
(defun my_list_without_last (lst)
	(if (null (cdr lst))
		Nil
		(cons (car lst) (my_list_without_last (cdr lst)))
	)
)

(defun my_member_before (elem lst)
	(if (my_member elem lst)
		(my_member_before elem (my_list_without_last lst))
		lst
	)
)

(defun my_remove_first (elem lst)
	(append 
		(my_member_before elem lst) 
		(cdr (my_member elem lst)))
)

(defun set-equal (set1 set2)
	(if (null set1)  
		(null set2)
		(if (my_member (car set1) set2) 
			(set-equal (cdr set1) (my_remove_first (car set1) set2))
			Nil
		)
	)
)

(set-equal '(1 2 3) '(1 2 3)) => T
(set-equal '() '()) => T
(set-equal '(1 2) '(1 2 3)) => NIL
(set-equal '(1) '(1 2)) => NIL
(set-equal '(1 2) '(1 (2))) => NIL
(set-equal '(1 (2)) '(1 (2))) => T
(set-equal '(1 (2)) '(1 (2 3))) => NIL
\end{lstlisting}


Комментарий к решению: данная функция корректно работает для сравнения как множеств, так и просто неупорядоченных совокупностей ((set-equal '(1 2 2 3) '(1 2 3)) => Nil).

\clearpage
\section*{3. Напишите свои необходимые функции, которые обрабатывают таблицу из 4-х точечных пар: (страна . столица), и возвращают по стране - столицу, а по столице — страну.}



\begin{lstlisting}[language=Lisp]

(defun find_capital_by_country (table country)
	(if (null table)
		(and (print '(No such country)) Nil)
		(if (eq (caar table) country)
			(cdar table)
			(find_capital_by_country (cdr table) country)
		)
	)
)

(defun find_country_by_capital (table capital)
	(if (null table)
		(and (print '(No such capital)) Nil)
		(if (eq (cdar table) capital)
			(caar table)
			(find_country_by_capital (cdr table) capital)
		)
	)
)

(defvar table)
(setq table '((Russia . Moscow) (GreatBritain . London)                  (France . Paris) (Italy . Roma)))

(find_capital_by_country table 'Russia) => MOSCOW
(find_capital_by_country table 'Italy) => ROMA
(find_capital_by_country table 'USA) =>"(NO SUCH COUNTRY)" NIL

(find_country_by_capital table 'Moscow) => Russia
(find_country_by_capital table 'Paris) => FRANCE
(find_country_by_capital table 'Washington) =>"(NO SUCH CAPITAL)" NIL

; вместо (caar table) можно использовать (row\_country (car table))
(defun row_country (row_table) (car row_table))
; вместо (cdar table) можно использовать (row\_capital (car table))
(defun row_capital (row_table) (cdr row_table))
; это сделает программу более гибкой при изменениях в формате таблицы, но медленней
\end{lstlisting}

\section*{4. Напишите функцию swap-first-last, которая переставляет в списке-аргументе первый и последний элементы.}

\begin{lstlisting}[language=Lisp]
; с использованиемм самостоятельно реализованных функций
(defun my_last (lst) 
	(if (null (cdr lst))
		(car lst)
		(my_last (cdr lst))
	)
)
	
(defun my_list_without_last (lst)
	(if (null (cdr lst))
		Nil
		(cons (car lst) (my_list_without_last (cdr lst)))
	)
)

(defun swap-first-last (lst)
	(if (cdr lst)
		(append 
			(cons (my_last lst) Nil)
			(my_list_without_last (cdr lst)) 
			(cons (car lst) Nil)	
		)
		lst
	)
)
(swap-first-last '(1 2 3)) => (3 2 1)
; для 2 приведенных ниже примеров и нужна проверка if (cdr lst) в swap-first-last
(swap-first-last '()) => Nil
(swap-first-last '(1)) => (1)
(swap-first-last '(1 2)) => (2 1)
(swap-first-last '((1 2) 3 4 5)) => (5 3 4 (1 2))

; с использованием встроенных функций (и только если последний элемент -- атом)
(defun swap-first-last (lst)
	(if (cdr lst)
		(let ((last_elem (car (last lst))))
			(append 
				(cons last_elem (remove last_elem (cdr lst))) 
				(cons (car lst) Nil))
		)
		lst
	)
)
\end{lstlisting}

\section*{5. Напишите функцию swap-two-element, которая переставляет в списке-аргументе два указанных своими порядковыми номерами элемента в этом списке.}

\begin{lstlisting}[language=Lisp]
; с использованиемм самостоятельно реализованных функций

; индексация начинается с 0
(defun my_nth (lst n)
	(if (or (< n 0) (< (my_length lst) (- n 1)))
		(print '(Error no such index))
		(if (= n 0)
			(car lst)
			(my_nth (cdr lst) (- n 1))
		)
	)
)

; from-элемента в результате нет
(defun list_slice_from (lst from)
	(if (= from -1)
		lst
		(list_slice_from (cdr lst) (- from 1))
	)
)

(defun list_without_last_n (lst n)
	(if (= n 0)
		lst
		(list_without_last_n (my_list_without_last lst) (- n 1))
	)
)

; to-элемента в результате нет
(defun list_slice_to (lst to)
	(list_without_last_n lst (- (my_length lst) to))
)

; элементы to и from не входят в результат
(defun list_slice_from_to (lst from to)
	(list_slice_from (list_slice_to lst to) from)
)






(defun swap-two-element-inner (lst index1 index2)
	(if (<= 0 index1 index2 (- (my_length lst) 1))
		(append 
			(list_slice_to lst index1)
			(cons (my_nth lst index2) Nil)
			(list_slice_from_to lst index1 index2)
			(if (= index1 index2) ;защита от дублирования
				Nil	
				(cons (my_nth lst index1) Nil)
			)
			(list_slice_from lst index2)
		)
		(print '(Error in indexes))
	)
)

(defun swap-two-element (lst index1 index2)
	(if (< index1 index2)
		(swap-two-element-inner lst index1 index2)
		(swap-two-element-inner lst index2 index1)
	)
)

(swap-two-element '(0 1 2 3 4 5 6 7) 2 4) => (0 1 4 3 2 5 6 7)
(swap-two-element '(0 1 2 3 4 5 6 7) 0 7) => (7 1 2 3 4 5 6 0)
(swap-two-element '(0 1 2 3 4 5 6 7) -1 2) => (ERROR IN INDEXES)
(swap-two-element '(0 1 2 3 4 5 6 7) 1 8) => (ERROR IN INDEXES)
(swap-two-element '(0 1 2 3 4 5 6 7) 5 3) => (0 1 2 5 4 3 6 7)
(swap-two-element '(0 1 2 3 4 5 6 7) 3 3) => (0 1 2 3 4 5 6 7)
\end{lstlisting}

\clearpage
\section*{6. Напишите две функции, swap-to-left и swap-to-right, которые производят одну круговую перестановку в списке-аргументе влево и вправо, соответственно}

\begin{lstlisting}[language=Lisp]
(defun my_list_without_last (lst)
	(if (null (cdr lst))
		Nil
		(cons (car lst) (my_list_without_last (cdr lst)))
	)
)

(defun my_last (lst) 
	(if (null (cdr lst))
		(car lst)
		(my_last (cdr lst))
	)
)

(defun swap-to-right (lst)
	(if (null (cdr lst))
		lst
		(cons (my_last lst) (my_list_without_last lst))
	)
)

(defun swap-to-left (lst)
	(if (null (cdr lst))
		lst
		(cons (cdr lst) (cons (car lst) Nil))
	)
)

(swap-to-left ()) => Nil
(swap-to-right ()) => Nil
(swap-to-left '(1)) => (1)
(swap-to-right '(1)) => (1)
(swap-to-left '(1 2 (3 4))) => ((2 (3 4)) 1)
(swap-to-right '(1 2 (3 4))) => ((3 4) 1 2)
\end{lstlisting}

\clearpage
\section*{7. Напишите функцию, которая добавляет к множеству двухэлементных списков новый двухэлементный список, если его там нет.}

\begin{lstlisting}[language=Lisp]
(defun my_equalp (x1 x2)
	(cond 
		((numberp x1) (and (numberp x2) (= x1 x2)))
		((atom x1) (and (atom x2) (eq x1 x2)))
		((consp x1) (and (consp x2)  (my_equalp (car x1)  (car x2)) (my_equalp (cdr x1) (cdr x2))))
		((listp x1) (and (listp x2) (= (my_length x1) (my_length x2)) (my_equalp (car x1)  (car x2)) (my_equalp (cdr x1) (cdr x2))))	
	)
)

(defun my_member (elem lst)
	(if (null lst)
		Nil
		(if (my_equalp elem (car lst))
			lst
			(my_member elem (cdr lst)))
	)
)

(defun update (elem set)
	(if (my_member elem set)
		set
		(cons elem set)
	)
)

(update '(0 1) '((1 2) (2 3) (3 4))) => ((0 1) (1 2) (2 3) (3 4))
(update '(1 2) '((1 2) (2 3) (3 4))) => ((1 2) (2 3) (3 4))
(update '(1 2) Nil) => ((1 2))
\end{lstlisting}

\clearpage
\section*{8. Напишите функцию, которая умножает на заданное число-аргумент первый числовой элемент списка из заданного 3-х элементного списка-аргумента, когда}

a) все элементы списка --- числа,

\begin{lstlisting}[language=Lisp]
(defun my_mult (num lst) 
	(cons 
		(* num (car lst)) 
		(cdr lst)
	)
)

(my_mult 2 '(1 2 3)) => (2 2 3)
\end{lstlisting}


б) элементы списка -- любые объекты.

\begin{lstlisting}[language=Lisp]

(defun my_mult (num smth)
	(cond 
		((null smth) Nil)
		((numberp smth) (* num smth))
		((atom smth) smth)
		; cons
		(T (let ((res_car (my_mult num (car smth))))
			(if (my_equalp (car smth) res_car)
				(cons (car smth) (my_mult num (cdr smth)))
				(cons res_car (cdr smth))
			)
		   )
		)
	)
)


(my_mult 2 '(1 2 3)) => (2 2 3)
(my_mult 2 '(a b c)) => (A B C)
(my_mult 2 '(a 2 3)) => (A 4 3)
(my_mult 2 '(a (2 . 3) (4 5 6))) => (A (4 . 3) (4 5 6))
(my_mult 2 '(a (b . 3) (4 5 6))) => (A (B . 6) (4 5 6))
(my_mult 2 '(a (b . c) (d 5 6))) => (A (B . C) (D 10 6))
(my_mult 2 '(a (b . c) (d e 6))) => (A (B . C) (D E 12))
(my_mult 2 '(a (b . c) (d f (g 5) 10))) => (A (B . C) (D F (G 10) 10))
(my_mult 2 Nil) => Nil
\end{lstlisting}

\section*{9. Напишите функцию, select-between, которая из списка-аргумента из 5 чисел выбирает только те, которые расположены между двумя указанными границами-аргументами и возвращает их в виде списка (упорядоченного по возрастанию списка чисел (+ 2 балла))}

\begin{lstlisting}[language=Lisp]
; без сортировки
(defun select-between (lst a b)
	(if (null lst)
		Nil
		(if (< a (car lst) b)
			(cons (car lst) (select-between (cdr lst) a b))
			(select-between (cdr lst) a b)
		)
	)
)

(select-between '(0 3 7 5 4) 1 6) => (3 5 4)

; с сортировкой
; блочная (карманная, корзинная) сортировка
(defun my_sort (lst)
	(if (null lst)
		Nil
		(append 
			(my_sort (remove-if-not (lambda (x) (> (car lst) x)) (cdr lst)))
			(remove-if-not (lambda (x) (= (car lst) x)) lst)
			(my_sort (remove-if-not (lambda (x) (< (car lst) x)) (cdr lst)))
		)
	)
)

; в функциях select-greater и select-lower предполагается, что переданный список 
;отсортирован по возрастанию
(defun select-greater (lst a)
	(if (null lst)
		Nil
		(if (< a (car lst))
			lst
			(select-greater (cdr lst) a)
		)
	)
)



(defun select-lower (lst b)
	(if (null lst)
		Nil
		(if (> b (car lst))
			(cons (car lst) (select-lower (cdr lst) b))
			Nil
		)
	)
)

(defun select-between-sorted (lst a b)	
	(select-greater (select-lower (my_sort lst) b) a)	
)

(select-between-sorted '(0 3 7 5 4) 1 6) => (3 4 5)

\end{lstlisting}

	\bibliographystyle{utf8gost705u}  % стилевой файл для оформления по ГОСТу
	
	\bibliography{51-biblio}          % имя библиографической базы (bib-файла)

	
\end{document}
