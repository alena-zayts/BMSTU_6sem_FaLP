\documentclass[12pt]{report}
\usepackage[utf8]{inputenc}
\usepackage[russian]{babel}
%\usepackage[14pt]{extsizes}
\usepackage{listings}
\usepackage{graphicx}
\usepackage{amsmath,amsfonts,amssymb,amsthm,mathtools} 
\usepackage{pgfplots}
\usepackage{filecontents}
\usepackage{float}
\usepackage{comment}
\usepackage{indentfirst}
\usepackage{eucal}
\usepackage{enumitem}
%s\documentclass[openany]{book}
\frenchspacing

\usepackage{indentfirst} % Красная строка

\usetikzlibrary{datavisualization}
\usetikzlibrary{datavisualization.formats.functions}

\usepackage{amsmath}


% Для листинга кода:
\lstset{ %
	language=c,                 % выбор языка для подсветки (здесь это С)
	basicstyle=\small\sffamily, % размер и начертание шрифта для подсветки кода
	numbers=left,               % где поставить нумерацию строк (слева\справа)
	numberstyle=\tiny,           % размер шрифта для номеров строк
	stepnumber=1,                   % размер шага между двумя номерами строк
	numbersep=5pt,                % как далеко отстоят номера строк от подсвечиваемого кода
	showspaces=false,            % показывать или нет пробелы специальными отступами
	showstringspaces=false,      % показывать или нет пробелы в строках
	showtabs=false,             % показывать или нет табуляцию в строках
	frame=single,              % рисовать рамку вокруг кода
	tabsize=2,                 % размер табуляции по умолчанию равен 2 пробелам
	captionpos=t,              % позиция заголовка вверху [t] или внизу [b] 
	breaklines=true,           % автоматически переносить строки (да\нет)
	breakatwhitespace=false, % переносить строки только если есть пробел
	escapeinside={\#*}{*)}   % если нужно добавить комментарии в коде
}


\usepackage[left=2cm,right=2cm, top=2cm,bottom=2cm,bindingoffset=0cm]{geometry}
% Для измененных титулов глав:
\usepackage{titlesec, blindtext, color} % подключаем нужные пакеты
\definecolor{gray75}{gray}{0.75} % определяем цвет
\newcommand{\hsp}{\hspace{20pt}} % длина линии в 20pt
% titleformat определяет стиль
\titleformat{\chapter}[hang]{\Huge\bfseries}{\thechapter\hsp\textcolor{gray75}{|}\hsp}{0pt}{\Huge\bfseries}


% plot
\usepackage{pgfplots}
\usepackage{filecontents}
\usetikzlibrary{datavisualization}
\usetikzlibrary{datavisualization.formats.functions}

\begin{document}
	%\def\chaptername{} % убирает "Глава"
	\thispagestyle{empty}
	\begin{titlepage}
		\noindent \begin{minipage}{0.15\textwidth}
			\includegraphics[width=\linewidth]{img/b_logo}
		\end{minipage}
		\noindent\begin{minipage}{0.9\textwidth}\centering
			\textbf{Министерство науки и высшего образования Российской Федерации}\\
			\textbf{Федеральное государственное бюджетное образовательное учреждение высшего образования}\\
			\textbf{~~~«Московский государственный технический университет имени Н.Э.~Баумана}\\
			\textbf{(национальный исследовательский университет)»}\\
			\textbf{(МГТУ им. Н.Э.~Баумана)}
		\end{minipage}
		
		\noindent\rule{18cm}{3pt}
		\newline\newline
		\noindent ФАКУЛЬТЕТ $\underline{\text{«Информатика и системы управления»}}$ \newline\newline
		\noindent КАФЕДРА $\underline{\text{«Программное обеспечение ЭВМ и информационные технологии»}}$\newline\newline\newline\newline\newline
		
		\begin{center}
			\noindent\begin{minipage}{1.1\textwidth}\centering
				\Large\textbf{  Отчет по лабораторной работе №}\newline
				\textbf{по дисциплине <<Функциональное и логическое}\newline
				\textbf{~~~программирование>>}\newline\newline
			\end{minipage}
		\end{center}
		
		\noindent\textbf{Тема} $\underline{\text{}}$\newline\newline
		\noindent\textbf{Студент} $\underline{\text{Зайцева А.А.~~~~~~~~~~~~~~~~~~~~~~~~~~~~~~~~~~~~~~~~~~~~~~~~~~~~~~~~~~~~~~~~~}}$\newline\newline
		\noindent\textbf{Группа} $\underline{\text{ИУ7-62Б~~~~~~~~~~~~~~~~~~~~~~~~~~~~~~~~~~~~~~~~~~~~~~~~~~~~~~~~~~~~~~~~~~~~~~~~~}}$\newline\newline
		\noindent\textbf{Оценка (баллы)} $\underline{\text{~~~~~~~~~~~~~~~~~~~~~~~~~~~~~~~~~~~~~~~~~~~~~~~~~~~~~~~~~~~~~~~~~~~~~~~~}}$\newline\newline
		\noindent\textbf{Преподаватель} $\underline{\text{Толпинская Н.Б., Строганов Ю. В.~~~~~~~~~~~~~~~~~~~~~~~~~~}}$\newline\newline\newline
		
		\begin{center}
			\vfill
			Москва~---~\the\year
			~г.
		\end{center}
	\end{titlepage}
	

\chapter*{Лабораторная работа №11}
\subsection*{Задание}
Разработать свою программу — <<Телефонный справочник>>. Протестировать работы программы.

\subsection*{Решение}
\begin{lstlisting}
domains
	surname = string.
	name = string.
	address = string.
	phone = integer.

predicates
	nondeterm record(name, surname, address, phone).

clauses
	record("Ivanov", "Ivan", "Moscow, Ivanovskay 5", 2294055).
	record("Petrov", "Petr", "Volgograd, Leninskaya 1", 8456372).
	record("Sidorov", "Ivan", "Ekaterinburg, Lesnaya 2/4", 8994527).
	record("Svidorov", "Pavel", "Moscow, Lesnaya 8", 8994558).
	record("Ivanov", "Ivan", "Ekaterinburg, Glavnaya 4", 6994566).

goal
	% record(Surname, _, _, Phone). =>
	% Surname=Ivanov, Phone=2294055
	% Surname=Petrov, Phone=8456372
	% Surname=Sidorov, Phone=8994527
	% Surname=Svidorov, Phone=8994558
	% Surname=Ivanov, Phone=6994566
	% 5 Solutions
	
	% record("Ivanov", "Ivan", "Moscow, Ivanovskay 5", Phone). => 
	% Phone=2294055
	% 1 Solution
	
	% record("Petrov", "Petr", "Volgograd, Leninskaya 2", Phone). =>
	% No Solution
	
	record(Surname, "Ivan", _, _). 
	% Surname=Ivanov
	% Surname=Sidorov
	% Surname=Ivanov
	% 3 Solutions
\end{lstlisting}

\chapter*{Лабораторная работа №12}
\section*{Постановка задачи}
Составить программу — базу знаний, с помощью которой можно определить, например, множество студентов, обучающихся в одном ВУЗе. Студент может одновременно обучаться в нескольких ВУЗах. Привести примеры возможных вариантов вопросов и варианты ответов (не менее 3-х), Описать порядок формирования вариантов ответа.

\subsection*{Решение}
\begin{lstlisting}
domains 
	name, surname, university, group = string
	course = integer

predicates
	student(name, surname, university, group, course)

clauses
	student("Alexey", "Romanov", "BMSTU", "IU7-63B", 3).
	student("Pavel", "Perestoronin", "BMSTU", "IU7-63B", 5).
	student("Ivan", "Cvetkov", "BMSTU", "IU7-43B", 2).
	student("Alexey", "Voyakin", "BMSTU", "IU7-64B", 3).
	student("Alexey", "Romanov", "MSU", "VMK-1", 1).
	student("Pavel", "Perestoronin", "MAI", "EM-8", 2).
	student("Ilya", "Bryukhov", "MAI", "ACS12", 2).
	student("Vladimir", "Nesterov", "MIPT", Group, Course) 
		:- student("Alexey", "Romanov", "BMSTU", Group, _),  student("Pavel", "Perestoronin", "BMSTU", Group, Course).

goal
	student(Name, Surname, "MIPT", _, _).
	student(Name, Surname, "BMSTU", _, 3).
	student("Alexey", \_, \_, "IU7-63B", \_)
\end{lstlisting}

Данная база знаний содержит информацию о студентах (имя, фамилия, вуз, группа, курс).\\

С помощью первого вопроса получаются все студенты , которые учатся в МФТИ. Происходит проход сверху вниз по всем фактам предиката \emph{student(name, surname, university, group, course)} и осуществляется унификация с \emph{study(Name, Surname, "BMSTU"{}, \_, \_)}. Унификацию успешно проходит один факт: \emph{study("Vladimir"{}, "Nesterov"{}, "BMSTU"{}, "IU7-63B"{}, 5)}.\\

С помощью второго вопроса получаются все студенты МГТУ, которые обучаются на 3 курсе.  Происходит проход по всем фактам предиката \emph{student(name, surname, university, group, course)} и осуществляется унификация с \emph{study(Name, Surname, "BMSTU"{}, \_, 3)}.  Успешно унификацию проходят факты \emph{study("Alexey"{}, "Romanov"{}, "BMSTU"{}, \_, 3)} и \emph{study("Alexey"{}, "Voyakin"{}, "BMSTU"{}, \_, 3)}.\\

С помощью второго вопроса получаются все Алексеи, которые обучаются в группе ИУ7-63Б.  Происходит проход по всем фактам предиката \emph{student(name, surname, university, group, course)} и осуществляется унификация с \emph{study(Name, \_, \_, "IU7-63B"{}, \_)}.  Успешно унификацию проходит \emph{study("Alexey"{}, "Romanov"{}, "BMSTU"{}, "IU7-63B"{}, 3)}.\\

\chapter*{Лабораторная работа №13}
\section*{Постановка задачи}
Составить программу, то есть модель предметной области — базу знаний, объединив в ней информацию — знания:

\begin{itemize}
	\item <<Телефонный справочник>>: фамилия, № телефона, адрес - структура (город, улица, № дома, № квартиры);
	\item <<Автомобили>>: фамилия владельца, марка, цвет, стоимость и др.;
	\item <<Вкладчики банков>>: фамилия, банк, счет, сумма и др.
\end{itemize}

Владелец может иметь несколько телефонов, автомобилией вкладов (Факты). Используя правила, обеспечить возможность поиска:

\begin{enumerate}
	\item \begin{itemize}
		\item по № телефона найти: фамилию, марку автомобиля, стоимость автомобиль (может быть несколько);
		\item используя сформированное в пункте А правило, по № телефона найти только марку автомобиля (автомобилей может быть несколько);
	\end{itemize}
	\item используя простой, не составной вопрос: по фамилии (уникальна в городе, но в разных городах есть однофамильцы) и городу проживания найти: улицу проживания, банки, в которых есть вклады и № телефона.
\end{enumerate}

Для одного из вариантов ответов, и для А, и для B, описать словесно порядок поиска ответа на вопрос, указав, как выбираются знания, и, при этом, для каждого этапа унификации, выписать подстановку — наибольший унификатор, и соответствующие примеры термов.

\subsection*{Решение}

\newpage
\begin{lstlisting}
domains
	surname, phone, city, street, house, apartment = string
	address = address(city, street, house, apartment)
	model, color, cost = string
	bank, sum = string

predicates
	man(surname, phone, address)
	car(surname, model, color, cost)
	deposit(surname, bank, sum)
	
	car_by_phone(phone, surname, model, cost)
	car_model_by_phone(phone, model)
	bank_and_street_by_surname_and_city(surname, city, bank, street)

clauses	
	man("Alexey", "89096412389", address("Krasnogorsk", "Lesnaya", "17", "5")).
	man("Vladimir", "890955550987", address("Moscow", "Sovetskaya", "134", "15")).
	man("Mikhail", "8100500321", address("Khimki", "Lesnaya", "27", "501")).
	man("Pavel", "87654329867", address("Moscow", "Tikhaya", "105", "52")).
	man("Alexey", "89096421389", address("Moscow", "Baumanskaya", "170", "1")).
	
	car("Alexey", "AMG", "Black", "100000").
	car("Vladimir", "Volvo", "Red", "50000").
	car("Pavel", "Nissan", "White", "709000").
	car("Mikhail", "Cadillac", "Black", "1000000").
	
	deposit("Alexey", "Tinkoff", "10000000000").
	deposit("Alexey", "Sber", "1000").
	deposit("Pavel", "Sber", "0").
	deposit("Mikhail", "Alpha", "10").
	
	car_by_phone(Phone, Surname, Model, Cost)
		:- man(Surname, Phone, _), car(Surname, Model, _, Cost).
	
	car_model_by_phone(Phone, Model) :- car_by_phone(Phone, _, Model, _).
	
	bank_and_street_by_surname_and_city(Surname, City, Bank, Street)
		:- man(Surname, _, address(City, Street, _, _)), deposit(Surname, Bank, _).

goal
	%car_by_phone("89096412389", Surname, Model, Cost).
	%car_model_by_phone("89096412389", Model).
	bank_and_street_by_surname_and_city("Alexey", "Moscow", Bank, Street).
\end{lstlisting}""\newline

\begin{comment}
Порядок формирования результата для 1-го вопроса:

\begin{table}[H]
	\begin{center}
		\begin{tabular}{|c c c |} 
			\hline
			№ шага & Сравниваемые термы; результаты; подстановка, если есть & Дальнейшие действия \\  
			\hline
			1 & Сравниваются & Прямой ход \\
			  & car\_by\_phone("89096412389", Surname, Model, Cost) & \\
			  & и car\_by\_phone(Phone, Surname, Model, Cost).  & \\
			  & Подстановка (Phone - "89096412389") &\\
			\hline
			2 & Сравниваются man(Surname, "89096412389"{}, \_) & Термы не \\
			  & и car\_by\_phone(Phone, Surname, Model, Cost). & унифицируемы. \\
			  & Они имеют разные функторы. &Переход к следующему \\
			  & & предложению\\
			\hline
			3 & Сравниваются man(Surname, "89096412389"{}, \_) & Термы не \\
			  & и car\_model\_by\_phone(Phone, Model). & унифицируемы. \\
			  & Они имеют разные функторы. & Переход к следующему \\
			  & & предложению\\
			\hline
			4 & Сравниваются man(Surname, "89096412389"{}, \_) & Термы не \\
			  & и man("Vladimir"{}, "890955550987"{}, address(. & унифицируемы. \\
		      & "Moscow"{}, "Sovetskaya"{}, "134"{}, "15"{})).  & Переход к следующему \\
			  & Термы не унифицируемы. & предложению. \\
			\hline
		    5 & Сравниваются man(Surname, "89096412389"{}, \_) & Термы не \\
			  & и man("Vladimir"{}, "8100500321"{}, address(. & унифицируемы. \\
			  & "Moscow"{}, "Sovetskaya"{}, "27"{}, "501"{})).  & Переход к следующему \\
			  & Термы не унифицируемы. & предложению. \\
			\hline
			6 & Сравниваются man(Surname, "89096412389"{}, \_) & Термы не \\
		      & и man("Vladimir"{}, "87654329867"{}, address(. & унифицируемы. \\
		   	  & "Moscow"{}, "Tikhaya"{}, "105"{}, "52"{})).  & Переход к следующему \\
			  & Термы не унифицируемы. & предложению. \\
			\hline
			7 & Сравниваются man(Surname, "89096412389"{}, \_) & Термы не \\
			  & и man("Vladimir"{}, "89096421381"{}, address(. & унифицируемы. \\
			  & "Moscow"{}, "Baumanskaya"{}, "170"{}, "1"{})).  & Переход к следующему \\
			  & Термы не унифицируемы. & предложению. \\
			\hline
			8 & Сравниваются man(Surname, "89096412389"{}, \_) & Прямой ход. \\
			  & и man("Alexey"{}, "89096412389"{}, address(. & Занесение \\
			  & "Krasnogorsk"{}, "Lesnaya"{}, "17"{}, "5"{})). Подстановка & Surname = "Alexey"{} \\
			  & (Surname = "Alexey", Phone = "89096412389"{}, & в ячейку\\
			  & \_ = address("Krasnogorsk"{}, "Lesnaya"{}, "17"{}, "5"{})) &\\
			\hline
			9 & Сравниваются car("Alexey"{}, Brand, \_, Cost) & Термы не \\
			  & и car\_by\_phone(Phone, Surname, Model, Cost). & унифицируемы. \\
			  & Они имеют разные функторы. &Переход к следующему \\
			  & & предложению\\
			\hline
			10 & Сравниваются car("Alexey"{}, Brand, \_, Cost) & Термы не \\
			  & и car\_model\_by\_phone(Phone, Model). & унифицируемы. \\
			  & Они имеют разные функторы. &Переход к следующему \\
			  & & предложению\\
			\hline
			11 & Сравниваются car("Alexey"{}, Brand, \_, Cost) & Термы не \\
			  & и man("Vladimir"{}, "890955550987"{}, address(. & унифицируемы. \\
			  & "Moscow"{}, "Sovetskaya"{}, "134"{}, "15"{})).  & Переход к следующему \\
			  & Разные функторы. & предложению. \\
			\hline
		\end{tabular}
	\end{center}
\end{table}

\begin{table}[H]
	\begin{center}
		\begin{tabular}{|c c c |} 
			\hline
			12 & Сравниваются car("Alexey"{}, Brand, \_, Cost) & Термы не \\
			  & и man("Vladimir"{}, "8100500321"{}, address(. & унифицируемы. \\
			  & "Moscow"{}, "Sovetskaya"{}, "27"{}, "501"{})).  & Переход к следующему \\
			  & Разные функторы. & предложению. \\
			\hline
			13 & Сравниваются car("Alexey"{}, Brand, \_, Cost) & Термы не \\
			  & и man("Vladimir"{}, "87654329867"{}, address(. & унифицируемы. \\
			  & "Moscow"{}, "Tikhaya"{}, "105"{}, "52"{})).  & Переход к следующему \\
			  & Разные функторы. & предложению. \\
			\hline
			14 & Сравниваются car("Alexey"{}, Brand, \_, Cost) & Термы не \\
			  & и man("Vladimir"{}, "89096421381"{}, address(. & унифицируемы. \\
			  & "Moscow"{}, "Baumanskaya"{}, "170"{}, "1"{})).  & Переход к следующему \\
			  & Разные функторы. & предложению. \\
			\hline
			15 & Сравниваются car("Alexey"{}, Brand, \_, Cost) & Термы не \\
			   & и man("Vladimir"{}, "89096421381"{}, address(. & унифицируемы. \\
			   & "Moscow"{}, "Baumanskaya"{}, "170"{}, "1"{})).  & Переход к следующему \\
		       & Разные функторы. & предложению. \\
			\hline
			16 & Сравниваются car("Alexey"{}, Brand, \_, Cost) & Термы не \\
			  & и car("Pavel"{}, "Nissan"{}, "White"{}, "709000"{}). & унифицируемы. \\
			  & Термы не унифицированы.  & Переход к следующему \\
			  &  & предложению. \\
			\hline
			17 & Сравниваются car("Alexey"{}, Brand, \_, Cost) & Термы не \\
			  & и car("Mikhail"{}, "Cadillac"{}, "Black"{}, "1000000"{}). & унифицируемы. \\
			  & Термы не унифицированы.  & Переход к следующему \\
			  &  & предложению. \\
			\hline
			18 & Сравниваются car("Alexey"{}, Brand, \_, Cost) & Прямой ход. Занесение \\
			  & и car("Alexey"{}, "AMG"{}, "Black"{}, "100000"{}). & Model = "AMG"{}, Cost = "100000"{}\\
			  & Подстановка (Model = "AMG, \_ = "Black",  & в результирующую ячейку. \\
			  & Cost = "100000")  & \\
			\hline
			19 & Результат: подстановка & \\
			   & (Name = "Alexey"{}, Model = "AMG"{}, Cost = "100000"{}) & \\
			\hline
		\end{tabular}
	\end{center}
\end{table}

Порядок формирования результата для 2-го вопроса:
\begin{table}[H]
	\begin{center}
		\begin{tabular}{|c c c |} 
			\hline
			№ шага & Сравниваемые термы; результаты; подстановка, если есть & Дальнейшие действия \\  
			\hline
			1 & Сравниваются & Прямой ход \\
			& car\_by\_phone(Phone, Surname, Model, Cost) & \\
			& и car\_by\_phone("89096412389", \_, Brand, \_).  & \\
			& Подстановка (Phone - "89096412389"{}) &\\
			\hline
			2 & Сравниваются man(Surname, "89096412389"{}, \_) & Термы не \\
			& и car\_by\_phone(Phone, Surname, Model, Cost). & унифицируемы. \\
			& Они имеют разные функторы. &Переход к следующему \\
			& & предложению\\
			\hline
			3 & Сравниваются man(Surname, "89096412389"{}, \_) & Термы не \\
			& и car\_model\_by\_phone(Phone, Model). & унифицируемы. \\
			& Они имеют разные функторы. & Переход к следующему \\
			& & предложению\\
			\hline
			4 & Сравниваются man(Surname, "89096412389"{}, \_) & Термы не \\
			& и man("Vladimir"{}, "890955550987"{}, address(. & унифицируемы. \\
			& "Moscow"{}, "Sovetskaya"{}, "134"{}, "15"{})).  & Переход к следующему \\
			& Термы не унифицируемы. & предложению. \\
			\hline
			5 & Сравниваются man(Surname, "89096412389"{}, \_) & Термы не \\
			& и man("Vladimir"{}, "8100500321"{}, address(. & унифицируемы. \\
			& "Moscow"{}, "Sovetskaya"{}, "27"{}, "501"{})).  & Переход к следующему \\
			& Термы не унифицируемы. & предложению. \\
			\hline
			6 & Сравниваются man(Surname, "89096412389"{}, \_) & Термы не \\
			& и man("Vladimir"{}, "87654329867"{}, address(. & унифицируемы. \\
			& "Moscow"{}, "Tikhaya"{}, "105"{}, "52"{})).  & Переход к следующему \\
			& Термы не унифицируемы. & предложению. \\
			\hline
			7 & Сравниваются man(Surname, "89096412389"{}, \_) & Термы не \\
			& и man("Vladimir"{}, "89096421381"{}, address(. & унифицируемы. \\
			& "Moscow"{}, "Baumanskaya"{}, "170"{}, "1"{})).  & Переход к следующему \\
			& Термы не унифицируемы. & предложению. \\
			\hline
			8 & Сравниваются man(Surname, "89096412389"{}, \_) & Прямой ход. \\
			& и man("Alexey"{}, "89096412389"{}, address(. & Занесение \\
			& "Krasnogorsk"{}, "Lesnaya"{}, "17"{}, "5"{})). Подстановка & Surname = "Alexey"{} \\
			& (Surname = "Alexey", Phone = "89096412389"{}, & в ячейку\\
			& \_ = address("Krasnogorsk"{}, "Lesnaya"{}, "17"{}, "5"{})) &\\
			\hline
			9 & Сравниваются car("Alexey"{}, Brand, \_, Cost) & Термы не \\
			& и car\_by\_phone(Phone, Surname, Model, Cost). & унифицируемы. \\
			& Они имеют разные функторы. &Переход к следующему \\
			& & предложению\\
			\hline
			10 & Сравниваются car("Alexey"{}, Brand, \_, Cost) & Термы не \\
			& и car\_model\_by\_phone(Phone, Model). & унифицируемы. \\
			& Они имеют разные функторы. &Переход к следующему \\
			& & предложению\\
			\hline
			11 & Сравниваются car("Alexey"{}, Brand, \_, Cost) & Термы не \\
			& и man("Vladimir"{}, "890955550987"{}, address(. & унифицируемы. \\
			& "Moscow"{}, "Sovetskaya"{}, "134"{}, "15"{})).  & Переход к следующему \\
			& Разные функторы. & предложению. \\
			\hline
		\end{tabular}
	\end{center}
\end{table}

\newpage

\begin{table}[H]
	\begin{center}
		\begin{tabular}{|c c c |} 
			\hline
			12 & Сравниваются car("Alexey"{}, Brand, \_, Cost) & Термы не \\
			& и man("Vladimir"{}, "8100500321"{}, address(. & унифицируемы. \\
			& "Moscow"{}, "Sovetskaya"{}, "27"{}, "501"{})).  & Переход к следующему \\
			& Разные функторы. & предложению. \\
			\hline
			13 & Сравниваются car("Alexey"{}, Brand, \_, Cost) & Термы не \\
			& и man("Vladimir"{}, "87654329867"{}, address(. & унифицируемы. \\
			& "Moscow"{}, "Tikhaya"{}, "105"{}, "52"{})).  & Переход к следующему \\
			& Разные функторы. & предложению. \\
			\hline
			14 & Сравниваются car("Alexey"{}, Brand, \_, Cost) & Термы не \\
			& и man("Vladimir"{}, "89096421381"{}, address(. & унифицируемы. \\
			& "Moscow"{}, "Baumanskaya"{}, "170"{}, "1"{})).  & Переход к следующему \\
			& Разные функторы. & предложению. \\
			\hline
			15 & Сравниваются car("Alexey"{}, Brand, \_, Cost) & Термы не \\
			& и man("Vladimir"{}, "89096421381"{}, address(. & унифицируемы. \\
			& "Moscow"{}, "Baumanskaya"{}, "170"{}, "1"{})).  & Переход к следующему \\
			& Разные функторы. & предложению. \\
			\hline
			16 & Сравниваются car("Alexey"{}, Brand, \_, Cost) & Термы не \\
			& и car("Pavel"{}, "Nissan"{}, "White"{}, "709000"{}). & унифицируемы. \\
			& Термы не унифицированы.  & Переход к следующему \\
			&  & предложению. \\
			\hline
			17 & Сравниваются car("Alexey"{}, Brand, \_, Cost) & Термы не \\
			& и car("Mikhail"{}, "Cadillac"{}, "Black"{}, "1000000"{}). & унифицируемы. \\
			& Термы не унифицированы.  & Переход к следующему \\
			&  & предложению. \\
			\hline
			18 & Сравниваются car("Alexey"{}, Brand, \_, Cost) & Прямой ход. Занесение \\
			& и car("Alexey"{}, "AMG"{}, "Black"{}, "100000"{}). & Model = "AMG"{} \\
			& Подстановка (Model = "AMG, \_ = "Black",  & в результирующую ячейку. \\
			& Cost = "100000")  & \\
			\hline
			19 & Результат: подстановка & \\
			& (Model = "AMG"{}) & \\
			\hline
		\end{tabular}
	\end{center}
\end{table}


Порядок формирования результата для 3-го вопроса:

\begin{table}[H]
	\begin{center}
		\begin{tabular}{|c c c |} 
			\hline
			№ шага & Сравниваемые термы; результаты; подстановка, если есть & Дальнейшие действия \\  
			\hline
			1 & Сравниваются & Термы не \\
			  & bank\_and\_street("Alexey"{}, "Moscow"{}, Bank) & унифицируемы. \\
			  & и car\_by\_phone(Phone, Surname, Model, Cost).  & Переход к \\
			  & Разные функторы. & следующему предложению.\\
			\hline
			2 & Сравниваются  & Прямой ход \\
			  & bank\_and\_street("Alexey"{}, "Moscow"{}, Bank, Street) & \\
			  & и bank\_and\_street(Name, City, Street, Bank, Street). & \\
			  & Подстановка (Name = "Alexey"{}, City = "Moscow"{}) &  \\
			\hline
			3 & Сравниваются man("Alexey"{}, Phone, address( & Термы не \\
			  & "Moscow"{}, Street, \_, \_)). & унифицируемы. \\
			  & и car\_by\_phone(Phone, Surname, Model, Cost). & Переход к следующему \\
			  & Разные функторы. & предложению\\
			\hline
			4 & Сравниваются man("Alexey"{}, Phone, address( & Термы не \\
		  	  & "Moscow"{}, Street, \_, \_)). & унифицируемы. \\
			  & и bank\_and\_street(Name, City, Street, Bank) & Переход к следующему \\
			  & Разные функторы. & предложению\\
			\hline
			5 & Сравниваются man("Alexey"{}, Phone, address( & Термы не \\
			  & "Moscow"{}, Street, \_, \_)). & унифицируемы. \\
			  & и man("Alexey"{}, "890955550987"{}, address("Moscow"{}, & Переход к следующему \\
			  & "Sovetskaya"{}, "134"{}, "15"{})). Термы не унифицируемы. & предложению. \\
			\hline
			6 & Сравниваются man("Alexey"{}, Phone, address( & Термы не \\
			  & "Moscow"{}, Street, \_, \_)). & унифицируемы. \\
			  & и man("Mikhail"{}, "8100500321"{}, address("Khimki"{}, & Переход к следующему \\
			  & "Lesnaya"{}, "27"{}, "501"{})). Термы не унифицируемы. & предложению. \\
			\hline
			7 & Сравниваются man("Alexey"{}, Phone, address( & Термы не \\
			  & "Moscow"{}, Street, \_, \_)). & унифицируемы. \\
			  & и man("Pavel"{}, "87654329867"{}, address("Moscow"{}, & Переход к следующему \\
			  & "Tikhaya"{}, "105"{}, "52"{})). Термы не унифицируемы. & предложению. \\
			\hline
			8 & Сравниваются man("Alexey"{}, Phone, address( & Термы не \\
			  & "Moscow"{}, Street, \_, \_)). & унифицируемы. \\
			  & и man("Alexey"{}, "89096412389"{}, address("Krasnogorsk"{}, & Переход к следующему \\
			  & "Lesnaya"{}, "17"{}, "5"{})). Термы не унифицируемы. & предложению. \\
			\hline
			9 & Сравниваются man("Alexey"{}, Phone, address( & Прямой ход, занесение\\
			  & "Moscow"{}, Street, \_, \_)). & Phone = "89096421381"{} \\
			  & и man("Alexey"{}, "89096421381"{}, address("Moscow"{}, & Street = "Baumanskaya"{} \\
			  & "Baumanskaya"{}, "170"{}, "1"{})). Подстановка & в результирующую ячейку \\
			  & (Phone = "89096421381"{}, Street = "Baumanskaya"{},)& \\
			  &  \_ = "170"{}, \_ = "34"{}) & \\
			\hline
			10 & Сравниваются deposit("Alexey"{}, Bank, \_) & Термы не \\
			& и car\_by\_phone(Phone, Surname, Model, Cost). & унифицируемы. \\
			& Разные функторы. &Переход к следующему \\
			& & предложению\\
			\hline
			11 & Сравниваются deposit("Alexey"{}, Bank, \_) & Термы не \\
			& и man("Alexey"{}, "890955550987"{}, address("Moscow"{}, & унифицируемы. \\
			& "Moscow"{}, "Sovetskaya"{}, "134"{}, "15"{})).  & Переход к следующему \\
			& "Sovetskaya"{}, "134"{}, "15"{})). Разные функторы & Предложению \\
			\hline
		\end{tabular}
	\end{center}
\end{table}


\begin{table}[H]
	\begin{center}
		\begin{tabular}{|c c c |} 
			\hline
			12 & Сравниваются deposit("Alexey"{}, Bank, \_) & Термы не унифицируемы. \\
			   & и man("Alexey"{}, "890955550987"{}, address("Moscow"{},  & Переход к следующему \\
			   & "Sovetskaya"{}, "134"{}, "15"{})). Разные функторы  & предложению \\
			\hline
			13 & Сравниваются deposit("Alexey"{}, Bank, \_) & Термы не унифицируемы. \\
			  & и man("Mikhail"{}, "8100500321"{}, address("Khimki"{}, & Переход к следующему \\
			  & "Lesnaya"{}, "27"{}, "501"{})). Разные функторы  & предложению. \\
			\hline
			14 & Сравниваются deposit("Alexey"{}, Bank, \_) & Термы не унифицируемы. \\
			  & и man("Pavel"{}, "87654329867"{}, address("Moscow"{}, & Переход к следующему \\
		  	  & "Tikhaya"{}, "105"{}, "52"{})). Разные функторы  & предложению. \\
			\hline
			15 & Сравниваются deposit("Alexey"{}, Bank, \_) & Термы не унифицируемы. \\
			  & и man("Alexey"{}, "89096412389"{}, address("Krasnogorsk"{}, & Переход к следующему \\
			  & "Lesnaya"{}, "17"{}, "5"{})). Разные функторы  & предложению. \\
			\hline
			16 & Сравниваются deposit("Alexey"{}, Bank, \_) & Термы не унифицируемы.\\
			  & и man("Alexey"{}, "89096421381"{}, address("Moscow"{}, & Переход к следующему  \\
			  & "Baumanskaya"{}, "170"{}, "1"{})). Разные функторы  & предложению \\
			\hline
		    17 & Сравниваются deposit("Alexey"{}, Bank, \_) & Термы не унифицируемы.\\
			   & и car("Pavel"{}, "Nissan"{}, "White"{}, "709000"{}) & Переход к следующему  \\
			   & Разные функторы. & предложению \\
			\hline
			18 & Сравниваются deposit("Alexey"{}, Bank, \_) & Термы не унифицируемы.\\
			   & и car("Mikhail"{}, "Cadillac"{}, "Black"{}, "1000000"{}) & Переход к следующему  \\
			   & Разные функторы. & предложению \\
			\hline
			19 & Сравниваются deposit("Alexey"{}, Bank, \_) & Термы не унифицируемы.\\
			   & и car("Alexey"{}, "AMG"{}, "Black"{}, "100000"{}) & Переход к следующему  \\
			   & Разные функторы. & предложению \\
			\hline
		    20 & Сравниваются deposit("Alexey"{}, Bank, \_) & Термы не унифицируемы.\\
			   & и deposit("Pavel"{}, "Sber"{}, "0"{}) & Переход к следующему  \\
			   & Термы не унифицируемы. & предложению \\
			\hline
			21 & Сравниваются deposit("Alexey"{}, Bank, \_) & Термы не унифицируемы.\\
			   & и deposit("Mikhail"{}, "Alpha"{}, "10"{}) & Переход к следующему  \\
			   & Термы не унифицируемы. & предложению \\
			\hline
			22 & Сравниваются deposit("Alexey"{}, Bank, \_) & Прямой ход\\
			   & и deposit("Alexey"{}, "Tinkoff"{}, "100500"{}). Подстановка & Занесение Bank = "Tinkoff"{} \\
			   & (Bank = "Tinkoff"{}, \_ = "100500"{}) & в результирующую ячейку. \\
			\hline
			23 & Результат: подстановка & \\
			& (Street = "Baumanskaya"{}, Bank = "Tinkoff"{})&\\
			\hline
		\end{tabular}
	\end{center}
\end{table}

\end{comment}




\chapter*{Теоретическая часть}

\section*{1. Что собой представляет программа на языке пролог?}


\textbf{Программа на Prolog} представляет собой базу знаний и вопрос. 

\begin{itemize}
	\item \textbf{База знаний} состоит из предложений -- фактов и правил, -- используя которые программа выдает ответ на вопрос. Каждое предложение  должно заканчиваться точкой.
	\begin{itemize}
		\item \textbf{Правило} имеет вид: A :- B1, ... , Bn, где A -- заголовок правила (составной терм, который содержит знание); B1, ... , Bn - тело правила (составные термы, которые содержат условия истинности этого знания), символ $":-"$ -- это специальный символ-разделитель.
		\item \textbf{Факт} -- это частный случай правила -- предложение, в котором отсутствует тело (то есть тело пустое).
	\end{itemize}
	\item \textbf{Вопрос} -- это частный случай правила -- предложение, которое состоит только из тела. Используется, чтобы определить, выполняется ли некоторое отношение между описанными в программе объектами. Система рассматривает вопрос как цель, к которой (к истинности которой) надо стремиться. Ответ на вопрос может оказаться логически положительным или отрицательным, в зависимости от того, может ли быть достигнута соответствующая цель.
\end{itemize}

Базис пролога -- матлогика. Используется символьная обработка,  декларативная методология.

\section*{2. Какова структура программы на Prolog?}

Программа на Prolog состоит из следующих разделов, каждый из которых начинается со своего заголовка.

\begin{itemize}
	\item директивы компилятора — зарезервированные символьные константы,
	\item CONSTANTS — раздел описания констант,
	\item DOMAINS — раздел описания доменов,
	\item DATABASE — раздел описания предикатов внутренней базы данных,
	\item PREDICATES — раздел описания предикатов,
	\item CLAUSES — раздел описания предложений базы знаний,
	\item GOAL — раздел описания внутренней цели (вопроса).
\end{itemize}

В программе не обязательно должны быть все разделы.

\section*{3. Как реализуется программа на Prolog? Как формируются результаты работы программы?}

Ответ: Программа на Prolog представляет собой базу знаний и вопрос. База знаний состоит из предложений -- фактов и правил, которые задают истинные знания. Ответ на вопрос может оказаться логически положительным или отрицательным, в зависимости от того, может ли быть достигнута соответствующая цель. 

Вопрос рассматривается системой как цель: найти возможность, исходя из базы знаний, ответить «Да» на поставленный вопрос . Вариантов ответить «Да» на может быть несколько. При поиске ответа рассматриваются альтернативные варианты и находятся все возможные решения (методом проб и ошибок) - множества значений переменных, при которых на поставленный вопрос можно ответить - «Да».


Для выполнения логического вывода используется механизм унификации, встроенный в систему.
Унификация – операция, которая позволяет формализовать процесс логического вывода. С практической точки зрения  - это основной вычислительный шаг, с помощью которого происходят:
\begin{itemize}
	\item двунаправленная передача параметров процедурам,
	\item неразрушающее присваивание,
	\item проверка условий (доказательство).
\end{itemize}

В процессе работы система выполняет большое число унификаций.  Попытка "увидеть одинаковость" – сопоставимость двух термов, может завершаться успехом или тупиковой ситуацией (неудачей). В последнем случае включается механизм отката к предыдущему шагу.

\section*{4. Что такое терм?}

Терм - основной элемент языка Prolog. Терм – это:

\begin{enumerate}
	\item константа (используется для обозначения объекта предметной области): 
	\begin{itemize}
		\item число (целое, вещественное),
		\item cимвольный атом -- комбинация символов латинского алфавита, цифр и ’\_’ (символа подчеркивания), начинающаяся со строчной буквы), используется для обозначения конкретного объекта предметной области или для обозначения конкретного отношения,
		\item строка -- последовательность символов, заключенных в кавычки;
	\end{itemize}
	\item переменная:
	\begin{itemize}
		\item именованная -- комбинация символов латинского алфавита, цифр и ’\_’, начинающаяся с прописной буквы или символа подчеркивания, может связываться с различными объектами (конкретизироваться),
		\item анонимная  - обозначается символом ’\_’, не может быть связана со значением;
	\end{itemize}
	\item составной терм -- средство фиксации информации о том, что между объектами существует определенная связь,  синтаксически представляется так: f(t1, t2, …, tm), где f -  функтор (символьная константа, обозначающая имя отношения между объектами), t1, t2, …, tm – термы (в том  числе  и составные), являющиеся аргументами (арность -- число аргументов).
\end{enumerate}

\section*{5. Что такое предикат в матлогике (математике)?}

Предикат в математической логике -- это (логическая) функция со множеством значений {0, 1} (истина/ложь), определенная на некотором множестве параметров. Предикат называю n-арным, если он определен на n-ой декартовой степени множества М. Таким образом, каждый набор параметров характеризуется либо как «истинный», либо как «ложный».

\section*{6. Что описывает предикат в Prolog?}

Процедура -- совокупность правил, описывающих определенное отношение (заголовки имеют одно и то же имя и одинаковую арность). Предикат -- отношение, определяемое процедурой. Таким образом, предикат в Prolog описывает отношение между аргументами процедуры. 

\section*{7. Назовите виды предложений в программе и приведите примеры таких предложений из вашей программы. Какие предложения являются основными, а какие - не основными? Каковы: синтаксис и семантика (формальный смысл) этих предложений (основных и неосновных)?}

В Prolog есть два типа предложений: 
\begin{itemize}
	\item \textbf{Правило} имеет вид: $A :- B1, ... , Bn$, где A -- заголовок правила (составной терм, который содержит знание); B1, ... , Bn - тело правила (составные термы, которые содержат условия истинности этого знания), символ $":-"$ -- это специальный символ-разделитель;
	\item \textbf{Факт} -- это частный случай правила -- в нем отсутствует тело (то есть тело пустое).
\end{itemize}



TODO 
Примеры из программы:
\begin{itemize}
	\item факт: \emph{car("Mikhail"{}, "Cadillac"{}, "Black"{}, "500000"). }
	\item правило: \emph{car\_by\_phone(Phone, Surname, Model, Cost) :- man(Surame, Phone, \_), car(Name, Model, \_, Cost). }
\end{itemize}



Основными называются предложения, не содержащие переменных. Они предназначены для описания отношений, формирования базы знаний. 

Предложения, содержащие переменные называются неосновными. Они предназначены для поиска ответа в базе знаний.



\section*{8. Каковы назначение, виды и особенности использования переменных в программе на Prolog? Какое предложение БЗ сформулировано в более общей - абстрактной форме: содержащее или не содержащее переменных?}

Переменные предназначены для передачи информации «во времени (за конечное число шагов получить результат) и пространстве (передача значений через параметры)», для повышения уровня абстракции. Они бывают именованными и анонимными. 

Именованная переменная представляет собой комбинацию символов латинского алфавита, цифр и ’\_’, начинающуюся с прописной буквы или символа подчеркивания. В процессе выполнения программы именованные переменные могут связываться с различными объектами – конкретизироваться. Именованная переменная является уникальной в рамках предложения. В разных предложениях может использоваться одно имя переменной для обозначения разных объектов.

Анонимная  переменная обозначается символом ’\_’. Она не может быть связана со значением. Любая анонимная переменная уникальна.



В момент фиксации система не знает, какой объект представляет переменная. Во многих языках последовательность действий при работе с переменными такая: задать значение переменной, а затем работать с самой переменной. В прологе же особый способ работы с переменными -- значение переменной не задается, система сама подбирает такое(-ие) значение(-я), чтобы условие было истинным.

Предложение БЗ, содержащее переменные, сформулировано в более общей форме, так как переменные не имеют значения и могут конкретизироваться различными объектами в ходе работы системы.




TODO??

\section*{9. Что такое подстановка?}

Подстановка - это функция, действующая из множества X переменных в множество T термов программы, (при этом каждой переменной $X_{i} \in  X$ ставится в соответствие терм $t_{i} \in  T$.



Пусть дан терм: $А(X_1, X_2,  \dots ,X_n)$.
Подстановка - множество пар вида: $\{X _ i = t _ i\}$, где $X_i$ –   переменная, а $t_i$ –  терм.


\section*{10. Что такое пример терма? Как и когда строится? Как Вы думаете система строит и хранит термы?}


Пусть $\omega: {X}_{1} = {t}_{1} , {X}_{2} = {t}_{2},… , {X}_{n} = {t}_{n}$ -- подстановка. Тогда результат применения подстановки к терму обозначается $A\omega$. Применение подстановки заключается в замене каждого вхождения переменной $x_i$ на соответствующий терм. Терм B называется примером терма A, если существует такая подстановка $\omega$, что $B = A\omega$.

В процессе выполнения программы система, используя встроенный алгоритм унификации, пытается обосновать возможность истинности вопроса, строя подстановки и примеры термов (вопроса и формулировки знания), используя базу знаний. Построение и подстановки производится путём конкретизации переменных. Сами термы хранятся в стеке.














\bibliographystyle{utf8gost705u}  % стилевой файл для оформления по ГОСТу
\bibliography{51-biblio}          % имя библиографической базы (bib-файла)
	
\end{document}
