\documentclass[12pt]{report}
\usepackage[utf8]{inputenc}
\usepackage[russian]{babel}
%\usepackage[14pt]{extsizes}
\usepackage{listings}
\usepackage{graphicx}
\usepackage{amsmath,amsfonts,amssymb,amsthm,mathtools} 
\usepackage{pgfplots}
\usepackage{filecontents}
\usepackage{float}
\usepackage{comment}
\usepackage{indentfirst}
\usepackage{eucal}
\usepackage{enumitem}
%s\documentclass[openany]{book}
\frenchspacing

\usepackage{indentfirst} % Красная строка

\usetikzlibrary{datavisualization}
\usetikzlibrary{datavisualization.formats.functions}

\usepackage{amsmath}


% Для листинга кода:
\lstset{ %
	language=c,                 % выбор языка для подсветки (здесь это С)
	basicstyle=\small\sffamily, % размер и начертание шрифта для подсветки кода
	numbers=left,               % где поставить нумерацию строк (слева\справа)
	numberstyle=\tiny,           % размер шрифта для номеров строк
	stepnumber=1,                   % размер шага между двумя номерами строк
	numbersep=5pt,                % как далеко отстоят номера строк от подсвечиваемого кода
	showspaces=false,            % показывать или нет пробелы специальными отступами
	showstringspaces=false,      % показывать или нет пробелы в строках
	showtabs=false,             % показывать или нет табуляцию в строках
	frame=single,              % рисовать рамку вокруг кода
	tabsize=2,                 % размер табуляции по умолчанию равен 2 пробелам
	captionpos=t,              % позиция заголовка вверху [t] или внизу [b] 
	breaklines=true,           % автоматически переносить строки (да\нет)
	breakatwhitespace=false, % переносить строки только если есть пробел
	escapeinside={\#*}{*)}   % если нужно добавить комментарии в коде
}


\usepackage[left=2cm,right=2cm, top=2cm,bottom=2cm,bindingoffset=0cm]{geometry}
% Для измененных титулов глав:
\usepackage{titlesec, blindtext, color} % подключаем нужные пакеты
\definecolor{gray75}{gray}{0.75} % определяем цвет
\newcommand{\hsp}{\hspace{20pt}} % длина линии в 20pt
% titleformat определяет стиль
\titleformat{\chapter}[hang]{\Huge\bfseries}{\thechapter\hsp\textcolor{gray75}{|}\hsp}{0pt}{\Huge\bfseries}


% plot
\usepackage{pgfplots}
\usepackage{filecontents}
\usetikzlibrary{datavisualization}
\usetikzlibrary{datavisualization.formats.functions}

\begin{document}
	%\def\chaptername{} % убирает "Глава"
	\thispagestyle{empty}
	\begin{titlepage}
		\noindent \begin{minipage}{0.15\textwidth}
			\includegraphics[width=\linewidth]{img/b_logo}
		\end{minipage}
		\noindent\begin{minipage}{0.9\textwidth}\centering
			\textbf{Министерство науки и высшего образования Российской Федерации}\\
			\textbf{Федеральное государственное бюджетное образовательное учреждение высшего образования}\\
			\textbf{~~~«Московский государственный технический университет имени Н.Э.~Баумана}\\
			\textbf{(национальный исследовательский университет)»}\\
			\textbf{(МГТУ им. Н.Э.~Баумана)}
		\end{minipage}
		
		\noindent\rule{18cm}{3pt}
		\newline\newline
		\noindent ФАКУЛЬТЕТ $\underline{\text{«Информатика и системы управления»}}$ \newline\newline
		\noindent КАФЕДРА $\underline{\text{«Программное обеспечение ЭВМ и информационные технологии»}}$\newline\newline\newline\newline\newline
		
		\begin{center}
			\noindent\begin{minipage}{1.1\textwidth}\centering
				\Large\textbf{Отчет по лабораторной работе №11}\newline
				\textbf{по дисциплине <<Функциональное и логическое}\newline
				\textbf{~~~программирование>>}\newline\newline
			\end{minipage}
		\end{center}
		
		\noindent\textbf{Тема} $\underline{\text{Среда Visual Prolog 5.2. Структура программы на Prolog}}$\newline\newline
		\noindent\textbf{Студент} $\underline{\text{Зайцева А.А.~~~~~~~~~~~~~~~~~~~~~~~~~~~~~~~~~~~~~~~~~~~~~~~~~~~~~~~~~~~~~~~~~}}$\newline\newline
		\noindent\textbf{Группа} $\underline{\text{ИУ7-62Б~~~~~~~~~~~~~~~~~~~~~~~~~~~~~~~~~~~~~~~~~~~~~~~~~~~~~~~~~~~~~~~~~~~~~~~~~}}$\newline\newline
		\noindent\textbf{Оценка (баллы)} $\underline{\text{~~~~~~~~~~~~~~~~~~~~~~~~~~~~~~~~~~~~~~~~~~~~~~~~~~~~~~~~~~~~~~~~~~~~~~~~}}$\newline\newline
		\noindent\textbf{Преподаватель} $\underline{\text{Толпинская Н.Б., Строганов Ю. В.~~~~~~~~~~~~~~~~~~~~~~~~~~}}$\newline\newline\newline
		
		\begin{center}
			\vfill
			Москва~---~\the\year
			~г.
		\end{center}
	\end{titlepage}
	

\section*{Задание 1}
 Разработать свою программу - <<Телефонный справочник>>. Протестировать работу программы.

\subsection*{Решение}
\begin{lstlisting}
domains
	surname = string.
	name = string.
	address = string.
	phone = integer.

predicates
	phone_record(name, surname, address, phone).

clauses
	phone_record("Ivanov", "Ivan", "Moscow, Ivanovskay 5", 2294055).
	phone_record("Petrov", "Petr", "Volgograd, Leninskaya 1", 8456372).
	phone_record("Sidorov", "Ivan", "Ekaterinburg, Lesnaya 2/4", 8994527).
	phone_record("Svidorov", "Pavel", "Moscow, Lesnaya 8", 8994558).
	phone_record("Ivanov", "Ivan", "Ekaterinburg, Glavnaya 4", 6994566).

goal
	phone_record("Ivanov", "Ivan", "Moscow, Ivanovskay 5", 2294055).
	% yes

	% phone_record(Surname, _, _, Phone). 
	% Surname=Ivanov, Phone=2294055
	% Surname=Petrov, Phone=8456372
	% Surname=Sidorov, Phone=8994527
	% Surname=Svidorov, Phone=8994558
	% Surname=Ivanov, Phone=6994566
	% 5 Solutions
	
	% phone_record("Ivanov", "Ivan", "Moscow, Ivanovskay 5", Phone). => 
	% Phone=2294055
	% 1 Solution
	
	% phone_record("Petrov", "Petr", "Volgograd, Leninskaya 2", Phone). =>
	% No Solution
	
	% phone_record(Surname, "Ivan", _, _). =>
	% Surname=Ivanov
	% Surname=Sidorov
	% Surname=Ivanov
	% 3 Solutions
\end{lstlisting}

\section*{Задание 2}

Составить программу – базу знаний, с помощью которой можно определить, например, множество студентов, обучающихся в одном ВУЗе и их телефоны. Студент НЕ может одновременно обучаться в нескольких ВУЗах. Привести примеры возможных вариантов  вопросов и варианты ответов (не менее 3-х). Описать порядок формирования вариантов  ответа.

Исходную базу знаний сформировать с помощью только фактов. 
*Исходную базу знаний сформировать, используя правила.
**Разработать свою базу знаний (содержание произвольно)

\subsection*{Решение}
\begin{lstlisting}
domains 
	name, surname, university, group = string
	course = integer

predicates
	student(name, surname, university, group, course)

clauses
	student("Alexey", "Romanov", "BMSTU", "IU7-63B", 3).
	student("Pavel", "Perestoronin", "BMSTU", "IU7-63B", 5).
	student("Ivan", "Cvetkov", "BMSTU", "IU7-43B", 2).
	student("Alexey", "Voyakin", "BMSTU", "IU7-64B", 3).
	student("Alexey", "Romanov", "MSU", "VMK-1", 1).
	student("Pavel", "Perestoronin", "MAI", "EM-8", 2).
	student("Ilya", "Bryukhov", "MAI", "ACS12", 2).
	student("Vladimir", "Nesterov", "MIPT", Group, Course) 
		:- student("Alexey", "Romanov", "BMSTU", Group, _),  student("Pavel", "Perestoronin", "BMSTU", Group, Course).

goal
	student(Name, Surname, "MIPT", _, _).
	student(Name, Surname, "BMSTU", _, 3).
	student("Alexey", \_, \_, "IU7-63B", \_)
\end{lstlisting}

Данная база знаний содержит информацию о студентах (имя, фамилия, вуз, группа, курс).\\

С помощью первого вопроса получаются все студенты , которые учатся в МФТИ. Происходит проход сверху вниз по всем фактам предиката \emph{student(name, surname, university, group, course)} и осуществляется унификация с \emph{study(Name, Surname, "BMSTU"{}, \_, \_)}. Унификацию успешно проходит один факт: \emph{study("Vladimir"{}, "Nesterov"{}, "BMSTU"{}, "IU7-63B"{}, 5)}.\\

С помощью второго вопроса получаются все студенты МГТУ, которые обучаются на 3 курсе.  Происходит проход по всем фактам предиката \emph{student(name, surname, university, group, course)} и осуществляется унификация с \emph{study(Name, Surname, "BMSTU"{}, \_, 3)}.  Успешно унификацию проходят факты \emph{study("Alexey"{}, "Romanov"{}, "BMSTU"{}, \_, 3)} и \emph{study("Alexey"{}, "Voyakin"{}, "BMSTU"{}, \_, 3)}.\\

С помощью второго вопроса получаются все Алексеи, которые обучаются в группе ИУ7-63Б.  Происходит проход по всем фактам предиката \emph{student(name, surname, university, group, course)} и осуществляется унификация с \emph{study(Name, \_, \_, "IU7-63B"{}, \_)}.  Успешно унификацию проходит \emph{study("Alexey"{}, "Romanov"{}, "BMSTU"{}, "IU7-63B"{}, 3)}.\\


















\bibliographystyle{utf8gost705u}  % стилевой файл для оформления по ГОСТу
\bibliography{51-biblio}          % имя библиографической базы (bib-файла)
	
\end{document}
