

\documentclass[12pt]{report}
\usepackage[utf8]{inputenc}
\usepackage[russian]{babel}
\usepackage[14pt]{extsizes}
\usepackage{listings}
\usepackage{graphicx}
\usepackage{amsmath,amsfonts,amssymb,amsthm,mathtools} 
\usepackage{pgfplots}
\usepackage{filecontents}
\usepackage{float}
\usepackage{indentfirst}
\usepackage{eucal}
\usepackage{enumitem}
%s\documentclass[openany]{book}
\frenchspacing

\usepackage{titlesec}
\titleformat{\section}
{\normalsize\bfseries}
{\thesection}
{1em}{}
\titlespacing*{\chapter}{0pt}{-30pt}{8pt}
\titlespacing*{\section}{\parindent}{*4}{*4}
\titlespacing*{\subsection}{\parindent}{*4}{*4}

\usepackage{indentfirst} % Красная строка

\usetikzlibrary{datavisualization}
\usetikzlibrary{datavisualization.formats.functions}

\usepackage{amsmath}

\usepackage{amssymb}

% Для листинга кода:
\lstset{ %
	language=lisp,                 % выбор языка для подсветки (здесь это С)
	texcl=true,
	extendedchars=\true,
	basicstyle=\small\sffamily, % размер и начертание шрифта для подсветки кода
	numbers=left,               % где поставить нумерацию строк (слева\справа)
	numberstyle=\tiny,           % размер шрифта для номеров строк
	stepnumber=1,                   % размер шага между двумя номерами строк
	numbersep=5pt,                % как далеко отстоят номера строк от подсвечиваемого кода
	showspaces=false,            % показывать или нет пробелы специальными отступами
	showstringspaces=false,      % показывать или нет пробелы в строках
	showtabs=false,             % показывать или нет табуляцию в строках
	frame=single,              % рисовать рамку вокруг кода
	tabsize=2,                 % размер табуляции по умолчанию равен 2 пробелам
	captionpos=t,              % позиция заголовка вверху [t] или внизу [b] 
	breaklines=true,           % автоматически переносить строки (да\нет)
	breakatwhitespace=false, % переносить строки только если есть пробел
	escapeinside={\#*}{*)},  % если нужно добавить комментарии в коде
	%inputencoding=utf8x,
	%extendedchars=\true
}



\usepackage[left=2cm,right=2cm, top=2cm,bottom=2cm,bindingoffset=0cm]{geometry}
% Для измененных титулов глав:
\usepackage{titlesec, blindtext, color} % подключаем нужные пакеты
\definecolor{gray75}{gray}{0.75} % определяем цвет
\newcommand{\hsp}{\hspace{20pt}} % длина линии в 20pt
% titleformat определяет стиль
\titleformat{\chapter}[hang]{\Huge\bfseries}{\thechapter\hsp\textcolor{gray75}{|}\hsp}{0pt}{\Huge\bfseries}


% plot
\usepackage{pgfplots}
\usepackage{filecontents}
\usepackage[unicode, pdftex]{hyperref}
\usetikzlibrary{datavisualization}
\usetikzlibrary{datavisualization.formats.functions}

 
\begin{document}



\section*{}
Конъюнкция, дизъюнкция, отрицание  --  базовые функции матлогики. Предикат - логическая функция.

Базис пролога -- матлогика.

Предикат -- логическая функция

Блоки:
База знаний --  clauses.
Запрос разработки -- goal.
Запросы могут быть конъ или дизъ, но нам будут запрещать их использовать

Терм -- константа, переменная или составное тело.

В прологе используется символьная обработка. Декларативная методология. Мы описываем систему знаний из предметной области. Потом задаем вопрос, но хотим получить не только да/нет, но и как (как побочный эффект)? Не запрещено использовать символы. 

Константы -- символьные атомы обозначение объекта/процесса предметной области -- комбинация латинских символов, начинающаяся с маленькой буквы.

А переменная - тоже комбинация символов. Начинается с большой латинской или с нижнего подчеркивания -- именованные переменные. Есть такде анонимные переменные, которые обозначаются одинаково \_

!!!Зачем нужны переменные -- для повышения уровня абстракции

Составные термы - зафиксировать информацию от том, что между какими-то объектами есть связь. f(t1, t2, ..., tn). f - главный функтор -- имя отношения между двумяя объектами, символьный атом (потому что могут быть внутри еще), t- терм.


student(ivanov, mgtu) - константы
student(X, mgtu) -- группа студентов из мгту

student(ivanov, mgtu) и student(ivanov) -- для системы разные запросы.

В момент фиксации система не знает, что такое X


Первые аргуметы считаются как объекты одной природы, вторые - другой. Только мы определям смысл.

Чем больше переменных, тем выше уровень абстракции.

База знаний состоит из фактов (без переменных -- основные, ост - неосн)
Правило
A:-B1,B, ...Bk
A -- заголовок правила (в заголовке формулирубтся знанния о томЮ что..) а тело B всезадает 
Заголовоок - фиксацияя знания о том, чтоо между аргументами мб истинная связь
student(X, mgtu):-докум(X, att), выше(ball, 296)

!!!Особенный сспособ работы с переменными. В ду=ругих языках задаем, потом работаем. Здесь не задаем значение, система сам подбирает значение переменной, чтобы услове былло истинным.  Цель системы -- ответить да

!!!Переменные нужны для передачи данных во времени и пространстве.  Во времени -- через несколько шаги получаем, в пространстве - через параметры, переменные.

запрограммированый метод резолюции позволяет делать поиск ответа на вопрос.























































































































































































































































































































































































































































































































































































































































































































































































































































































































































\end{document}


